% /*@@
%   @file      Preface.tex
%   @date      Sun Jul 20 11:41:11 CEST 2003
%   @author    Jonathan Thornburg, borrowing heavily from the
%              Preface for the Cactus User's Guide
%   @desc 
%   Preface for the Cactus Reference Manual
%   @enddesc 
%   @version $Header$
% @@*/

%%%%%%%%%%%%%%%%%%%%%%%%%%%%%%%%%%%%%%%%%%%%%%%%%%%%%%%%%%%%%%%%%%%%%%%
%%%%%%%%%%%%%%%%%%%%%%%%%%%%%%%%%%%%%%%%%%%%%%%%%%%%%%%%%%%%%%%%%%%%%%%

{\large \bf Preface} 
\label{sec:pr}
 
\vskip .5cm

This document will eventually be a complete reference manual for
the Cactus Code. However, it is currently under
development, so please be patient if you can't find what you need.
Please report omissions, errors, or suggestions to 
and of our contact addresses below, and we will
try and fix them as soon as possible. 

\vskip .5cm

{\bf Overview of documentation}

\vskip .5cm

This guide covers the following topics

\begin{Lentry}

\item [{\bf Part~\ref{part:CCTKReference}: {\tt CCTK\_*} Function Reference.}]
	Here all the \verb|CCTK_*()| Cactus flesh functions
	which are available to thorn writers are described.

\item [{\bf Part~\ref{part:UtilReference}: {\tt Util\_*} Function Reference.}]
	Here all the \verb|Util_*()| Cactus flesh functions
	which are available to thorn writers are described.

\end{Lentry}

Other topics to be discussed in separate documents include:

\begin{Lentry}

\item [{\bf Users' Guide}] This gives a general overview of the
  Cactus Computational Tool Kit, including overall design/architecture,
  how to get/configure/compile/run it, and general discussions of the
  how to program in Cactus.

\item [{\bf Relativity Thorn Guide}] This will contain details about the arrangements and thorns making up the Cactus Relativity Tool Kit, one of the major 
 motivators, and still the driving force, for the Cactus Code.

\item [{\bf Flesh Maintainers Guide}] 
 This will contain all the gruesome details
 about the inner workings of Cactus, for all those who want or need to 
 expand or maintain the core of Cactus.

\end{Lentry}

\vskip .5cm

{\bf Typographical Conventions}

\begin{Lentry}

\item[{\tt Typewriter}] Is currently used for everything you type,
	for program names, and code extracts.
\item[{\tt < ... >}] Indicates a compulsory argument.
\item[{\tt [ ... ]}] Indicates an optional argument.
\item[{\tt |}] Indicates an exclusive or.

\end{Lentry}
 
\vskip .5cm

{\bf How to Contact Us}

\vskip .5cm

Please let us know of any errors or omissions in this guide, as well
as suggestions for future editions. These can be reported via our 
bug tracking system at \url{http://www.cactuscode.org}, or via email to
{\tt cactusmaint@cactuscode.org}. Alternatively, write to us at

\vskip .5cm
The Cactus Team\\
Center for Computation \& Technology\\
216 Johnston Hall\\
Louisiana State University\\
Baton Rouge, LA 70803\\
USA


\vskip .5cm

{\bf Acknowledgements}

\vskip .5cm

Hearty thanks to all those who have helped with documentation for the
Cactus Code. Special thanks to those who struggled with the earliest
sparse versions of this guide and sent in mistakes and suggestions,
in particular John Baker, Carsten Gundlach, Ginny Hudak-David, 
Sai Iyer, Paul Lamping, Nancy Tran and Ed Seidel. 
