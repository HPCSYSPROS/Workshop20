% *======================================================================*
%  Cactus Thorn template for ThornGuide documentation
%  Author: Ian Kelley
%  Date: Sun Jun 02, 2002
%  $Header$
%
%  Thorn documentation in the latex file doc/documentation.tex
%  will be included in ThornGuides built with the Cactus make system.
%  The scripts employed by the make system automatically include
%  pages about variables, parameters and scheduling parsed from the
%  relevant thorn CCL files.
%
%  This template contains guidelines which help to assure that your
%  documentation will be correctly added to ThornGuides. More
%  information is available in the Cactus UsersGuide.
%
%  Guidelines:
%   - Do not change anything before the line
%       % START CACTUS THORNGUIDE",
%     except for filling in the title, author, date, etc. fields.
%        - Each of these fields should only be on ONE line.
%        - Author names should be separated with a \\ or a comma.
%   - You can define your own macros, but they must appear after
%     the START CACTUS THORNGUIDE line, and must not redefine standard
%     latex commands.
%   - To avoid name clashes with other thorns, 'labels', 'citations',
%     'references', and 'image' names should conform to the following
%     convention:
%       ARRANGEMENT_THORN_LABEL
%     For example, an image wave.eps in the arrangement CactusWave and
%     thorn WaveToyC should be renamed to CactusWave_WaveToyC_wave.eps
%   - Graphics should only be included using the graphicx package.
%     More specifically, with the "\includegraphics" command.  Do
%     not specify any graphic file extensions in your .tex file. This
%     will allow us to create a PDF version of the ThornGuide
%     via pdflatex.
%   - References should be included with the latex "\bibitem" command.
%   - Use \begin{abstract}...\end{abstract} instead of \abstract{...}
%   - Do not use \appendix, instead include any appendices you need as
%     standard sections.
%   - For the benefit of our Perl scripts, and for future extensions,
%     please use simple latex.
%
% *======================================================================*
%
% Example of including a graphic image:
%    \begin{figure}[ht]
% 	\begin{center}
%    	   \includegraphics[width=6cm]{MyArrangement_MyThorn_MyFigure}
% 	\end{center}
% 	\caption{Illustration of this and that}
% 	\label{MyArrangement_MyThorn_MyLabel}
%    \end{figure}
%
% Example of using a label:
%   \label{MyArrangement_MyThorn_MyLabel}
%
% Example of a citation:
%    \cite{MyArrangement_MyThorn_Author99}
%
% Example of including a reference
%   \bibitem{MyArrangement_MyThorn_Author99}
%   {J. Author, {\em The Title of the Book, Journal, or periodical}, 1 (1999),
%   1--16. {\tt http://www.nowhere.com/}}
%
% *======================================================================*

% If you are using CVS use this line to give version information
% $Header$

\documentclass{article}

% Use the Cactus ThornGuide style file
% (Automatically used from Cactus distribution, if you have a
%  thorn without the Cactus Flesh download this from the Cactus
%  homepage at www.cactuscode.org)
\usepackage{../../../../doc/latex/cactus}

\begin{document}

% The author of the documentation
\author{Peter Diener \textless diener@cct.lsu.edu\textgreater}

% The title of the document (not necessarily the name of the Thorn)
\title{Summation By Parts}

% the date your document was last changed, if your document is in CVS,
% please use:
\date{$ $Date$ $}

\maketitle

% Do not delete next line
% START CACTUS THORNGUIDE

% Add all definitions used in this documentation here
%   \def\mydef etc

% Add an abstract for this thorn's documentation
\begin{abstract}
Calculate first derivates of grid functions using finite difference stencils
that satisfy summation by parts.
\end{abstract}

% The following sections are suggestive only.
% Remove them or add your own.

\section{Introduction}
Given a discretization $x_0\ldots x_N$ of a computational domain $x\in[a,b]$
with gridspacing $h$ a one dimensional finite difference operator
approximation to a first derivative, $D$, is said to satisfy summation by
parts (SBP) with respect to a scalar product (defined by its
coefficients $\sigma_{ij}$)
\begin{equation}
\langle u, v\rangle_h = h \sum_{i=0}^{N} u_i v_j \sigma_{ij}
\end{equation}
if the property
\begin{equation}
\langle u, Dv\rangle_h +\langle Du, v\rangle_h = \left . (uv)\right|^b_a
\end{equation}
is satisfied for all possible gridfunctions $u$ and $v$. 

At a given finite difference order, there are several different ways of doing
this depending on the structure of the scalar product. The three commonly
considered cases are the diagonal norm, the restricted full norm and the full
norm (see figure~\ref{fig:norm} for the structure).
\begin{figure}[t]
\[
\sigma = \left ( \begin{array}{cccccc}
           x &   &   &   &   & \\
             & x &   &   &   & \\
             &   & x &   &   & \\
             &   &   & x &   & \\
             &   &   &   & 1 & \\
             &   &   &   &   & \ddots
         \end{array} \right ),\mbox{\hspace{0.2em}}
\sigma = \left ( \begin{array}{ccccccc}
           x &   &   &   &   &   & \\
             & x & x & x & x &   & \\
             & x & x & x & x &   & \\
             & x & x & x & x &   & \\
             & x & x & x & x &   & \\
             &   &   &   &   & 1 & \\
             &   &   &   &   &   & \ddots
        \end{array} \right ),\mbox{\hspace{0.2em}}
\sigma = \left ( \begin{array}{cccccc}
           x & x & x & x &   & \\
           x & x & x & x &   & \\
           x & x & x & x &   & \\
           x & x & x & x &   & \\
             &   &   &   & 1 & \\
             &   &   &   &   & \ddots
        \end{array} \right )
\]
\caption{The structure of the scalar product matrix in the diagonal case
(left), the restricted full case (middle) and the full case (right) for the
4th order interior operators. Only non-zero elements are shown.}
\label{fig:norm}
\end{figure}

In the following we denote the order of accuracy at the boundary by $\tau$, 
the order in the interior by $s$ and the width of the boundary 
region\footnote{The width of the region where standard centered finite
differences are not used.} by $r$.

For the diagonal norm case it turns out that with $r=2\tau$ it is possible
to find SBP operators with $s=2\tau$ (at least when $\tau\le 4$), i.e.\ the
order of accuracy at the boundary is half the order in the interior. For the
restricted full norm case $\tau = s-1$ when $r=\tau +2$ and for the full norm
case $\tau = s-1$ when $r=\tau + 1$.

The operators are named after their norm and their interior and boundary
orders. Thus, for example, we talk about diagonal norm 6-3 operators and
restricted full norm 4-3 operators.

In the diagonal case the 2-1 and 4-3 operators are unique whereas the 6-3
and 8-4 operators have 1 and 3 free parameters, respectively. In the restricted
full norm case the 4-3 operators have 3 free parameters and the 6-5 operators
have 4, while in the full norm case the number of free parameters is 1 less
than the restricted full case.
\section{Numerical Implementation}
Currently this thorn implements only diagonal and restricted full norm SBP
operators. The diagonal norm 2-1, 4-2 and 6-3 and the restricted full norm
4-3 are the ones listed in \cite{strand93} where in the presence of free
parameters the set of parameters giving a minimal bandwidth have been chosen.
For the diagonal norm 8-4 case the minimal bandwidth choice are to restrictive
with respect to the Courant factor and in this case parameters are chosen
so as to maximize the Courant factor\footnote{This is done by choosing
parameters that minimizes the largest eigenvalue of the amplification matrix
for a simple 1D advection equation with a penalty method implementation of
periodic boundary conditions as suggested in \cite{lrt05}.}. In addition the
restricted full norm 6-5 operator has been implemented. This was calculated
using a Mathematica script kindly provided by Jos\'{e} M.
Mart\'{\i}n-Garc\'{\i}a.

For the restricted full norm 6-5 SBP operators a compatible dissipation
operator has been implemented as well based on \cite{msn04}.
\section{Using This Thorn}
The basic functionality of calculating finite difference approximations are
performed by a set of aliased functions. In this way things are kept flexible
in the sense that different thorns each can provide their own way of
calculating derivatives but if they use the same calling interface the
user can switch between different derivative schemes by simply activating
the appropriate finite differencing thorn. Also the same derivative routines
can be called from both C and Fortran.

There are currently two different calling interfaces corresponding to different
levels of flexibility. In the first case the fact that derivatives are
approximated by finite differencing is completely hidden for the user. Using
this interface the same physics thorn could in principle be used with a
finite difference driver, a finite elements driver or a spectral driver. There
is however a price to pay for this flexibility due to the fact that the
operations are performed on whole grid functions at a time so that storage
for the derivatives has to be provided. This can significantly increase the
amount of memery required for the evolution. The alternative interface
returns instead the finite difference coefficients and allows the user to
calculate derivatives on a pointwise basis. This can significantly save
memory but limits the physics thorn to use finite differences.
\subsection{Obtaining This Thorn}
The thorn is currently still in development and so is not generally available.
Access can be requested by contacting the thorn maintainer.
\subsection{Basic Usage}
In order to use the derivative routines the appropriate declarations have 
to be added to your thorns interface.ccl file. To use the interface that
works on whole grid functions you have to add:
\begin{verbatim}
SUBROUTINE Diff_gv ( CCTK_POINTER_TO_CONST IN cctkGH, \
                     CCTK_INT IN dir, \
                     CCTK_REAL IN ARRAY var, \
                     CCTK_REAL OUT ARRAY dvar, \
                     CCTK_INT IN table_handle )
USES FUNCTION Diff_gv
\end{verbatim}
Here {\tt cctkGH} is the pointer to the cactus GH, {\tt dir} is the direction
of the derivative (0 for $x$, 1 for $y$ and 2 for $z$), {\tt var} is the grid
function to calculate derivatives of and {\tt dvar} is the grid function where
the derivative is to be stored and {\tt table\_handle} is an integer that
contains a handle for a keyword table with optional parameters. If there are
no optional paramters just pass in a negative value.

To use the interface that returns the finite difference coefficients you have 
to add:
\begin{verbatim}
SUBROUTINE Diff_coeff ( CCTK_POINTER_TO_CONST IN cctkGH, \
                        CCTK_INT IN dir, \
                        CCTK_INT IN nsize, \
                        CCTK_INT OUT ARRAY imin, \
                        CCTK_INT OUT ARRAY imax, \
                        CCTK_REAL OUT ARRAY q, \
                        CCTK_INT IN table_handle )
\end{verbatim}
Here, again, {\tt cctkGH} is the pointer to the cactus GH, {\tt dir} is the
direction of the derivative, {\tt nsize} is the grid size in the direction
you are interested in, {\tt imin} and {\tt imax} are 1D integer arrays of size
{\tt nsize} that on return will contain the minimum and maximum index of the
stencil in direction {\tt dir}, {\tt q} is a 2D real array of size 
{\tt nsize}$\times${\tt nsize} that on return will contain the coefficients
for the derivatives. The {\tt table\_handle} serves the same purpose as in
the previous case.

%\subsection{Special Behaviour}

%\subsection{Interaction With Other Thorns}

\subsection{Examples}
The following piece of Fortran code, shows how to calculate derivatives of
a grid function, f, and store the derivatives in the x-, y- and z-directions
in the grid functions  dxf, dyf and dzf:
\begin{verbatim}
  call Diff_gv (cctkGH, 0, f, dxf, -1)
  call Diff_gv (cctkGH, 1, f, dyf, -1)
  call Diff_gv (cctkGH, 2, f, dzf, -1)
\end{verbatim}
In order to use the interface for doing the pointwise derivatives, the
following Fortran90 example shows the necessary declarations and an example
of how to calculate the derivatives:
\begin{verbatim}
  CCTK_REAL, dimension(:,:), allocatable :: qx, qy, qz
  CCTK_INT, dimension(:), allocatable :: iminx, imaxx, iminy, &
                                         imaxy, iminz, imaxz
  CCTK_REAL :: idelx, idely, idelz
  CCTK_INT :: i, j, k

  allocate ( qx(cctk_lsh(1),cctk_lsh(1)), &
             qy(cctk_lsh(2),cctk_lsh(2)), &
             qz(cctk_lsh(3),cctk_lsh(3)), &
             iminx(cctk_lsh(1)), imaxx(cctk_lsh(1)), &
             iminy(cctk_lsh(2)), imaxy(cctk_lsh(2)), &
             iminz(cctk_lsh(3)), imaxz(cctk_lsh(3)) )

  call Diff_Coeff ( cctkGH, 0, cctk_lsh(1), iminx, imaxx, qx, -1 )
  call Diff_Coeff ( cctkGH, 1, cctk_lsh(2), iminy, imaxy, qy, -1 )
  call Diff_Coeff ( cctkGH, 2, cctk_lsh(3), iminz, imaxz, qz, -1 )

  idelx = 1.0d0 / CCTK_DELTA_SPACE(1)
  idely = 1.0d0 / CCTK_DELTA_SPACE(2)
  idelz = 1.0d0 / CCTK_DELTA_SPACE(3)

  do k = 1, cctk_lsh(3)
    do j = 1, cctk_lsh(2)
      do i = 1, cctk_lsh(1)
        dxf(i,j,k) = idelx * sum ( qx(iminx(i):imaxx(i),i) * &
                                     f(iminx(i):imaxx(i),j,k) )
        dyf(i,j,k) = idely * sum ( qy(iminy(j):imaxy(j),j) * &
                                     f(i,iminy(j):imaxy(j),k) )
        dzf(i,j,k) = idelz * sum ( qz(iminz(k):imaxz(k),k) * &
                                     f(i,j,iminz(k):imaxz(k)) )

      end do
    end do
  end do

  deallocate ( qx, qy, qz, iminx, imaxx, iminy, imaxy, iminz, imaxz )
\end{verbatim}
\subsection{Support and Feedback}
This thorn is maintained by Peter Diener. Any questions and comments should
be directed by e-mail to diener@cct.lsu.edu.

\section{History}
This thorn grew out of the needs of a multipatch relativity code and as
such was initially designed to those needs. The addition of the possibility
of passing in a handle for a keyword table was done at the request of
Jonathan Thornburg and should make it easy to extend the thorn with
additional features. The addition of the coefficient interface was done
at the request of Bela Szilagyi who felt that the storage overhead was
to large for his code. In the near future the thorn will be extended with
second derivatives SBP operators.

%\subsection{Thorn Source Code}
%
%\subsection{Thorn Documentation}

\section{Acknowledgements}
We thank Jos\'{e} M.\ Mart\'{\i}n-Garc\'{\i}a for very kindly providing us
with a very well designed mathematica script to calculate the finite
difference and scalar product coefficients.

\begin{thebibliography}{9}
\bibitem{strand93} Bo Strand, 1994, Journal of Computational Physics, 110,
47--67.
\bibitem{lrt05} Luis Lehner, Oscar Reula and Manuel Tiglio, in preparation.
\bibitem{msn04} Ken Mattsson, Magnus Sv\"{a}rd and Jan Nordstr\"{o}m, 2004,
Journal of Scientific Computing, 21, 57--79.

\end{thebibliography}

% Do not delete next line
% END CACTUS THORNGUIDE

\end{document}
