% *======================================================================*
%  Cactus Thorn template for ThornGuide documentation
%  Author: Ian Kelley
%  Date: Sun Jun 02, 2002
%  $Header: /numrelcvs/AEIDevelopment/DriftCorrect4/doc/documentation.tex,v 1.2 2005/05/29 09:09:21 schnetter Exp $                                                             
%
%  Thorn documentation in the latex file doc/documentation.tex 
%  will be included in ThornGuides built with the Cactus make system.
%  The scripts employed by the make system automatically include 
%  pages about variables, parameters and scheduling parsed from the 
%  relevant thorn CCL files.
%  
%  This template contains guidelines which help to assure that your     
%  documentation will be correctly added to ThornGuides. More 
%  information is available in the Cactus UsersGuide.
%                                                    
%  Guidelines:
%   - Do not change anything before the line
%       % START CACTUS THORNGUIDE",
%     except for filling in the title, author, date, etc. fields.
%        - Each of these fields should only be on ONE line.
%        - Author names should be separated with a \\ or a comma.
%   - You can define your own macros, but they must appear after
%     the START CACTUS THORNGUIDE line, and must not redefine standard 
%     latex commands.
%   - To avoid name clashes with other thorns, 'labels', 'citations', 
%     'references', and 'image' names should conform to the following 
%     convention:          
%       ARRANGEMENT_THORN_LABEL
%     For example, an image wave.eps in the arrangement CactusWave and 
%     thorn WaveToyC should be renamed to CactusWave_WaveToyC_wave.eps
%   - Graphics should only be included using the graphicx package. 
%     More specifically, with the "\includegraphics" command.  Do
%     not specify any graphic file extensions in your .tex file. This 
%     will allow us to create a PDF version of the ThornGuide
%     via pdflatex.
%   - References should be included with the latex "\bibitem" command. 
%   - Use \begin{abstract}...\end{abstract} instead of \abstract{...}
%   - Do not use \appendix, instead include any appendices you need as 
%     standard sections. 
%   - For the benefit of our Perl scripts, and for future extensions, 
%     please use simple latex.     
%
% *======================================================================* 
% 
% Example of including a graphic image:
%    \begin{figure}[ht]
% 	\begin{center}
%    	   \includegraphics[width=6cm]{MyArrangement_MyThorn_MyFigure}
% 	\end{center}
% 	\caption{Illustration of this and that}
% 	\label{MyArrangement_MyThorn_MyLabel}
%    \end{figure}
%
% Example of using a label:
%   \label{MyArrangement_MyThorn_MyLabel}
%
% Example of a citation:
%    \cite{MyArrangement_MyThorn_Author99}
%
% Example of including a reference
%   \bibitem{MyArrangement_MyThorn_Author99}
%   {J. Author, {\em The Title of the Book, Journal, or periodical}, 1 (1999), 
%   1--16. {\tt http://www.nowhere.com/}}
%
% *======================================================================* 

% If you are using CVS use this line to give version information
% $Header: /numrelcvs/AEIDevelopment/DriftCorrect4/doc/documentation.tex,v 1.2 2005/05/29 09:09:21 schnetter Exp $

\documentclass{article}

% Use the Cactus ThornGuide style file
% (Automatically used from Cactus distribution, if you have a 
%  thorn without the Cactus Flesh download this from the Cactus
%  homepage at www.cactuscode.org)
\usepackage{../../../../doc/latex/cactus}

\begin{document}

% The author of the documentation
\author{Erik Schnetter \textless schnetter@aei.mpg.de\textgreater}

% The title of the document (not necessarily the name of the Thorn)
\title{DriftCorrect4}

% the date your document was last changed, if your document is in CVS, 
% please use:
\date{$ $Date: 2005/05/29 09:09:21 $ $}

\maketitle

% Do not delete next line
% START CACTUS THORNGUIDE

% Add all definitions used in this documentation here 
%   \def\mydef etc

% Add an abstract for this thorn's documentation
\begin{abstract}

\end{abstract}

% The following sections are suggestive only.
% Remove them or add your own.

\section{Introduction}

(to be written)



\section{Radial drift correction}

There is a new parameter, \texttt{radial\_correction\_method}, which
determines the modification that is applied to radial components of
the shift when \texttt{do\_radial\_correction} is set.

The default value of this parameter is \texttt{linear}, and
corresponds to the previous behaviour whereby the correction increases
linearly with radius from the origin.

An alternate method, \texttt{annular}, has been introduced which
corresponds roughly to the correction used by Br\"ugmann, Tichy,
Jansen (2004).  The radial rdot is attenuated by a factor
\begin{equation}
  detg^n * (c^2 + 1)^s / (\rho_0 (c^2 + \rho^2/\rho_0^2))^s
\end{equation}
where
\begin{description}
\item[$detg$]
        is the determinant of the 3-metric
\item[$n$, $s$]
        are exponents determined respectively by the parameters
        \texttt{attenuation\_n} and \texttt{attenuation\_s}
\item[$c$]
        s a constant determined by the parameter \texttt{attenuation\_c}
\item[$\rho$]
        is the radial distance of a point from the $z$-axis
\item[$\rho_0$]
        is the centre of the adjustment
\end{description}

The effect of these adjustments is to restrict the adjustment to an
annular region centred around some radius $\rho_0$ from the $z$-axis.
The value of $\rho_0$ is determined by the \texttt{position\_x} and
\texttt{position\_y} parameters to this thorn, which determine the
desired location of the black hole.  The shape and falloff can be
controlled by the parameters \texttt{c} and \texttt{s}.  BTJ (2004)
suggest values of \texttt{c=1} and \texttt{s=2}.

I've modified the BTJ correction by the $detg^n$ factor in front.
This is an experiment to see whether this can help to remove the drift
correction during the merger phase: For positive values of $n$,
degenerate values of $detg$ (which might be expected when the BHs are
coming close together in physical space but not in coordinate space)
will reduce the effect of the correction.  This is currently being
tested. The default value, $n=0$, reproduces the BTJ-style
attenuation.



\begin{thebibliography}{9}

\end{thebibliography}

% Do not delete next line
% END CACTUS THORNGUIDE

\end{document}
