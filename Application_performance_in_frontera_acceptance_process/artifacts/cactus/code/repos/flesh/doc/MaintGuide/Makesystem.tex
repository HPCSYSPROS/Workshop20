% /*@@
%   @file      Makesystem.tex
%   @date      Wed Jan 12 14:38:29 2000
%   @author    Tom Goodale
%   @desc 
%   
%   @enddesc 
%   @version $Header$
% @@*/

\begin{cactuspart}{3}{The Make System}{}{$Revision$}
\renewcommand{\thepage}{\Alph{part}\arabic{page}}

\chapter{Introduction}

The make system has several design criteria:

\begin{itemize}
\item
Must be able to build the code on all supported platforms.
\item
Must allow object files and executables for different architectures to 
co-exist without conflicts.
\item{}
Must allow object files and exectutables for different compiler options to
co-exist without conflict.
\item{}
Must allow object files and executables for different thornsets to 
co-exist without conflicts.
\item{}
Thorn-writers must be able to easily add and remove files from their
thorns without having to edit files provided by the flesh or other thorns.
\item{}
Thorn-writers must be able to control the compilation options and 
dependencies of their own thorns.
\end{itemize}

The first criterion is achieved by standarising to the use of the freely
available GNU make programme, which is available on all platforms, and
the use of {\em Autoconf} to detect the specific features of the particular
machine the user is compiling on.

The next three criteria are achieved by the introduction of {\em configurations}
which contain all the information and object files associated with a 
particular combination of machine, compilation options and thorns.

The final criteria are achieved by allowing the thorn-writer to specify their
files and options in configuration files readable by the GNU make program,
or by specifying their own make file.

\section{Note on philosophy of the make system}

Make options can be divided into two classes.

\begin{itemize}
\item{Configuration-time options}
Things which have an effect on the resulting executable. 
E.g. optimisation or debugging options.
\item{Make-time options}
Things which don't effect the final executable.
E.g. warning-flags, flags to make in parallel.
\end{itemize}

Whenever an option is added to the make system care should be taken
to preserve this distinction.  It should be possible to go to the
{\em config-data} directory of a configuration and examine the files
there to determine how an executable was built.  It should not be
necessary to know the command-line used to build it.


%%%%%%%%%%%%%%%%%%%%%%%%%%%%%%%%%%%%%%%%%%%%%%%%%%%%%%%%%%%%%%%%%%%%%%%%%%%%%
\chapter{Make files}
\label{ch:makefiles}

\section{Makefile}
\label{sec:makefiles:Makefile}

This is the master makefile.  

In normal operation it calls {\em make.configuration} with the
-j TJOBS flag to build TJOBS thorns in parallel.

\section{lib/make/make.configuration}
\label{sec:makefiles:configuration}

This is the makefile which actually builds a configuration.

All built objects for a configuration go into a subdirectory called
{\em build} of the configuration.

For each thorn listed in the {\em make.thornlist} file generated
by the {\em CST} it runs {\em make.thornlib} or a file called {\em makefile} 
in the thorn's own source directory to generate a library for
that thorn.  Before running the sub-makefile it changes directory to a subdirectory
of the {\em build} directory with the same name as the thorn and sets

\begin{itemize}
\item{TOP}
The CCTK top-level directory.
\item{SRCDIR}
The thorn's source directory.
\item{CONFIG}
The {\em config} subdirectory of the configuration. 
\item{NAME}
The name of the library which should be created for the thorn
(including directory info).
\item{THORN}
The name of the thorn.
\end{itemize}

The sub-makefile is passed the -j FJOBS flag to build FJOBS files in parallel.

If {\em make.thornlist} doesn't exist, it runs the {\em CST} to generate it from
the {\em ThornList} file.

If {\em ThornList} doesn't exist, it generates a list of thorns in the 
{\em arrangements} and then gives the user the option to edit the
list.

\section{lib/make/make.thornlib}
\label{sec:makefiles:thornlib}

This makefile is responsible for producing a library from the contents of a
thorn's source directory.

In each source directory of a thorn the author may put two files.

\begin{itemize}
\item{make.code.defn}
This should contain a line
\begin{verbatim}
SRCS = 
\end{verbatim}
which lists the names of source files {\em in that directory}.
\item{make.code.deps}
This is an optional file which gives standard make dependency rules
for the files in that directory.
\end{itemize}

In addition the thorn's top-level {\em make.code.defn} file can
contain a line
\begin{verbatim}
SUBDIRS = 
\end{verbatim}

which lists {\em all} subdirectories of the thorn's {\em src} directory which
contain files to be compiled.

To process the subdirectories the makefile runs the sub-makefile {\em make.subdir}
in each subdirectory. 

Once that is done it compiles files in the {\em src} directory and then all
the object files into a library which is named by the {\em NAME} make-variable.

All object files are compiled by the rules given in {\em make.config.rules}.

Since the make language doesn't contain looping constructions it is a bit tricky
to construct the full list of object files.  To do this the makefile uses
the GNU make {\em foreach} function to include all the subdirectory 
{\em make.code.defn} files, along with two auxiliary files {\em make.pre} and
{\em make.post} which are included respectively before and after each 
{\em make.code.defn} file.  These auxiliary files allow the {\em SRCS} variables set in
the {\em make.code.defn} files to be concatanated onto one make variable 
{\em CCTK\_SRCS}.

Extensive use is made of the two different flavours of variable (simply-expanded and 
recursively-expanded) available within GNU make.

The GNU `-include' construct is used to suppress warnings when an optional file is
not available.

\section{lib/make/make.subdir}
\label{sec:makefiles:subdir}

This builds all the object files for a specific subdirectory according
to the list of files provided by the {\em make.code.defn} file
in that subdirectory.  Extra dependencies can be provided for these 
files by the presence of the optional file {\em make.code.deps} in the
directory.

\section{lib/make/make.pre and lib/make/make.post}
\label{sec:makefiles:prepost}

These are auxiliary files used to construct the full list of source
files for a particular thorn.

{\em make.pre} resets the {\em SRCS} variable to an empty value.  
{\em make.post} adds the name of the subdirectory onto all filenames
in the {\em SRCS} variable and adds the resulting list to the 
{\em CCTK\_SRCS} make variable.

%%%%%%%%%%%%%%%%%%%%%%%%%%%%%%%%%%%%%%%%%%%%%%%%%%%%%%%%%%%%%%%%%%%%%%%%%%%%%
\chapter{Autoconf stuff}
\label{ch:autoconf}

GNU autoconf is a program designed to detect the features available
on a particular platform.  It can be used to determine the 
compilers available on a platform, what the CPU and operating 
system are, what flags the compilers take, and as many other things
as m4 macros can be written to cover.

Autoconf is configured by a file {\em configure.in} which autoconf
turns into a file called {\em configure} (which should never be
editted by hand).  The cactus configuration includes the resulting
{\em configure} file and this should not need to be regenerated by
other than flesh-maintainers.

When the {\em configure} script is run it takes the files
{\em config.h.in}, {\em make.config.defn.in}, {\em make.config.rules.in},
and {\em make.config.deps.in} and generates new files in the 
configuration's {\em config-data} subdirectory with the same names with the
{\em .in} stripped off.  The configure script replaces certain parts of these
files with the values it has detected for this architecture.

In addition {\em configure} runs the {\em configure.pl} perl script to
do things which can only be done easily by perl.

\section{configure.in}
\label{sec:autoconf:configure}

This and the macro-definition file {\em aclocal.m4} are the 
sources for the {\em configure} script.  Autoconf should be
run in this directory if either of these files is editted.

Once the script has determined the host architecture, it checks
the {\em known-architecture} directory for any preferred 
compilers.  By default autoconf macros will choose GNU CC if
it is available, however for some architectures this may not be
desirable.

It then proceeds to determine the available compilers and auxiliary
programs if they haven't alrady been specified in an environment variable
or in the {\em known-architecture} file for this architecture.

Once the set of programs to be used has been detected or chosen, the
known-architecture files are again checked for specific features
which would otherwise require the writing of complicated macros to detect.
(Remember that the goal is that people don't need to write autoconf macros
or run autoconf themselves.)

Once that is done it looks at each subdirectory of the {\em extras}
directory for packages which have their own configuration process.  If a
subdirectory has an exectubale file called {\em setup.sh} this is called.

The rest of the script is concerned with detecting various header files,
sizes of various types, and of setting defaults for things which haven't
been set by the {\em known-architecture} file.

\section{config.h.in}
\label{sec:autoconf:h}

This file is turned into {\em config.h} in the {\em config-data}
directory in the configuration.

It contains C preprocessor macros which define various features or
configuration options.

\section{make.config.defn.in}
\label{sec:autoconf:defn}

This file is turned into {\em make.config.defn} in the {\em config-data}
directory in the configuration.

It contains make macros needed to define or build the particular 
configuration.

\section{make.config.rules.in}
\label{sec:autoconf:rules}

This file is turned into {\em make.config.rules} in the {\em config-data}
directory in the configuration.

It contains the rules needed to create an object file from a source file.
Note that currently this isn't modified by the configuration process, as
everything is controlled from variables in the {\em make.config.defn} file.
However this situation may change in the future if really necessary.

\section{make.config.deps.in}
\label{sec:autoconf:deps}

This file is turned into {\em make.config.deps} in the {\em config-data}
directory in the configuration.

Currently this file is empty;  it may gain content later if we need to
use autoconf to generate dependency stuff.

\section{aclocal.m4}
\label{sec:autoconf:aclocal}

This contains m4 macros not distributed with autconf.

\section{CCTK\_functions.sh}
\label{sec:autoconf:functions}

This contains Bourne-shell functions which can be used by 
the configure script, or by stuff in the {\em extras} or 
{\em known-architrectures} subdirectories.

\subsection{CCTK\_Search}

This can be used to search a set of directories for a specific file
or files and then set a variable as a result.

Usage: CCTK\_Search $<$variable$>$ $<$subdirectories to search$>$ $<$what to search for$>$ [base directory]

It will search each of the listed subdirectories of the base directory for the desired
file or directory, and, if it's found, set the variable to the name of the subdirectory.

\subsection{CCTK\_CreateFile}

Creates a file with specific contents.

Usage: CCTK\_CreateFile $<$filename$>$ $<$content$>$ 

Note that this can only put one line in the file to begin with.  Additional
lines can be added with {\em CCTK\_WriteLine}.

\subsection{CCTK\_WriteLine}

Write one line to a file.

Usage: CCTK\_WriteLine $<$file$>$ $<$line$>$

\section{known-architectures}
\label{sec:autoconf:knownarch}

This contains files which tell autoconf about specific not-easily-detectable
features about particular architectures.  Each file in this directory is
named with the name held by the host\_os autoconf variable.

Each file is called twice by the configure script.  Once to determine
the `preferred-compilers' for that architecture, and once for
everything else.

The first time is straight after the operating system is 
determined, and the variable {\em CCTK\_CONFIG\_STAGE} is set to
`preferred-compilers'.  It should only set the names of compilers,
and not touch anything else.  

The second time it is called is after the compilers and
auxiliary programs have been detected or otherwise chosen. 
{\em CCTK\_CONFIG\_STAGE} is set to `misc' in this case.  This stage
can be used to set compiler options based upon the chosen compilers.
The scripts are allowed to write (append) to {\em cctk\_archdefs.h} in
this stage if it needs to add to the C preprocessor macros included
by the code.  {\em CCTK\_WriteLine} can be used to write to the
file.

\section{extras}
\label{sec:autoconf:extras}

This directory is used to hold configuration programs for optional
extra packages.

If a subdirectory of this directory contains an executable file 
{\em setup.sh}, this file is run.

The two files {\em cctk\_extradefs.h} and {make.extra.defn} can be appended
to, to add c preprocessor macros or add/modify make variables respectively.
{\em CCTK\_WriteLine} can be used to do this.  

Note that include directories should be added to {\em SYS\_INC\_DIRS} and not
directly to {\em INC\_DIRS}.

\section{config.sub and config.guess}
\label{sec:autoconf:subguess}

These files are provided in the autoconf distribution.  They are used to
determine the host operating system, cpu, etc and put them into a
canonical form.

The files distributed with Cactus are slightly modified to allow recognition
of the Cray T3E, to work with the Portland compilers under Linux, and to
not do something stupid with unrecognised HP machines.

%%%%%%%%%%%%%%%%%%%%%%%%%%%%%%%%%%%%%%%%%%%%%%%%%%%%%%%%%%%%%%%%%%%%%%%%%%%
\chapter{Perl scripts}
\label{ch:perlscripts}

Various perl scripts are used in the make system.

\section{setup\_configuration.pl}
\label{sec:perlscripts:setup}

This is called by the top level makefile to create a new configuration or
to modify an old one.  It parses an options file setting environment 
variables as specified in that file, and then runs the autoconf-generated
configure script.

\section{configure.pl}
\label{sec:perlscripts:configure}

This file is called from the configure script to determine the way Fortran
names are represented for the specified Fortran compiler.  It works out the
names for subroutines/functions, and for common blocks, and writes a
perl script which can be used to convert a name to the appropriate form so
C and Fortran can be linked together.

\section{new\_thorn.pl}
\label{sec:perlscripts:newthorn}

This generates the skeleton for a new thorn.

%%%%%%%%%%%%%%%%%%%%%%%%%%%%%%%%%%%%%%%%%%%%%%%%%%%%%%%%%%%%%%%%%%%%%%%%%%
\end{cactuspart}
