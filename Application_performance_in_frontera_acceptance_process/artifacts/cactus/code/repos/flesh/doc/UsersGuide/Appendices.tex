% /*@@
%   @file      Appendices.tex
%   @date      27 Jan 1999
%   @author    Tom Goodale, Gabrielle Allen, Gerd Lanferman, Thomas Radke
%   @desc
%              Appendices for the Cactus User's Guide
%   @enddesc
%   @version   $Header$
% @@*/

\begin{cactuspart}{Appendices}{}{$Revision$}
\label{part:Appendices}
\renewcommand{\thepage}{\Alph{part}\arabic{page}}

%%%%%%%%%%%%%%%%%%%%%%%%%%%%%%%%%%%%%%%%%%%%%%%%%%%%%%%%%%%%%%%%%%%%%%%%%%%%%%%%
%%%%%%%%%%%%%%%%%%%%%%%%%%%%%%%%%%%%%%%%%%%%%%%%%%%%%%%%%%%%%%%%%%%%%%%%%%%%%%%%
%%%%%%%%%%%%%%%%%%%%%%%%%%%%%%%%%%%%%%%%%%%%%%%%%%%%%%%%%%%%%%%%%%%%%%%%%%%%%%%%

\chapter{Glossary}
\label{sec:glossary}

\begin{Lentry}

\item[alias function]
  See \textit{function aliasing}.
\item[AMR]
  \textit{Automatic Mesh Refinement}
\item[analysis]
\item[API]
  \textit{Applications Programming Interface}, the interface provided by
  some software component to programmers who use the component. 
  An API usually consists of subroutine/function calls, but may also include
  structure definitions and definition of constant values.
  The Cactus Reference Manual documents most of the Cactus flesh APIs.
\item[arrangement]
  A collection of thorns, stored in a subdirectory of the Cactus
  \verb|arrangements| directory.  See Section~\ref{sec:arrangements}.
\item[autoconf]
  A GNU program which builds a configuration script which can be used
  to make a Makefile.
\item[boundary zone]
  A boundary zone is a set of points at the edge of a grid, interpreted as
  the boundary of the physical problem, and which contains boundary data,
  e.g.\ Dirichlet conditions or von Neumann conditions.
  (See also \textit{symmetry zone}, \textit{ghost zone}.)
\item[Cactus]
Distinctive and unusual plant, which is adapted to extremely arid and hot environments, showing a wide range of anatomical and physiological features which conserve water. Cacti stems have expanded into green succulent structures containing the chlorophyll necessary for life and growth, while the leaves have become the spines for which cacti are so well known.\footnote{%%%
http://en.wikipedia.org/wiki/Cactus
       }%%%
\item[CCTK]
  \textit{Cactus Computational Tool Kit} (The Cactus flesh and computational
  thorns).
\item[CCL]
  The \textit{Cactus Configuration Language}, this is the language that
  the thorn configuration files are written in.
  See Section~\ref{sec:Appendix.ccl}.
\item[configuration]
  The combination of a set of thorns, and all the Cactus configure
  options which affect what binary will be produced when compiling
  Cactus.  For example, the choice of compilers (Cactus \verb|CC|,
  \verb|CXX|, \verb|CUCC|, and \verb|F90| configure options) and the
  compiler optimization settings (\verb|OPTIMISE|/\verb|OPTIMIZE| and
  \verb|*_OPTIMISE_FLAGS| configure options) are part of a
  configuration (these flags change what binary is produced), but the
  Cactus \verb|VERBOSE| and \verb|WARN| configure options aren't part
  of a configuration (they don't change what binary will be produced).
  See Section~\ref{sec:configurations}.
\item[checkout]
  Get a copy of source code from SVN.  See Section~\ref{sec:checkout}.
\item[checkpoint]
  Save the entire state of a Cactus run to a file, so that the run can be
  restarted at a later time.
  See Sections~\ref{sec:checkpointing}, \ref{chap:cp_recovery_methods}.
\item[computational grid]
  A discrete finite set of spatial points in $\Re^n$
  (typically, $1 \le n \le 3$).
  Historically, Cactus has required these points to be uniformly spaced
  (uniformly spaced grid), but now, Cactus
  supports non-uniform spacings (non-uniformly spaced grid), and mesh refinement.

  The grid consists of the physical domain and the boundary and symmetry
  points.

  See \textit{grid functions} for the typical use of grid points.
\item[convergence]
  Important, but often neglected.
\item[CST]
  The \textit{Cactus Specification Tool}, which is the set of Perl
  scripts which parse the thorns' \texttt{.ccl} files, and generates the
  code that binds the thorn source files with the flesh.
\item[SVN]
  \textit{Subversion} is the favoured code
  distribution system for Cactus.
  See Sections~\ref{sec:checkout},\ref{sec:Appendix.svn}.
\item[domain decomposition]
  The technique of breaking up a large computational problem into parts
  that are easier to solve.  In Cactus, it refers especially to a decomposition
  wherein the parts are solved in parallel on separate computer processors.
\item[driver]
  A special kind of thorn which creates and handles grid hierarchies 
  and grid variables.
  Drivers are responsible for memory management for grid variables,
  and for all parallel operations, in response to requests from the
  scheduler.
  See Section~\ref{sec:parallelisation}.
\item[evolution]
  An iteration interpreted as a step through time.  Also, a particular Cactus
  schedule bin for executing routines when evolution occurs.
\item[flesh]
  The Cactus routines which hold the thorns together, allowing them to 
  communicate and scheduling things to happen with them. This is what you
  get if you check out Cactus from our SVN repository.
\item[friend] Interfaces that are \textit{friends}, share their collective
  set of protected grid variables.
  See Section~\ref{sec:Appendix.interface}~\ref{subsec:interface_ccl}.
\item[function aliasing]
  The process of referring to a function to be provided by an interface
  independently of which thorn actually contains the function, or what
  language the function is written in.  The function is called an
  \textit{alias function}.  See Section~\ref{sec:function_aliasing},
  \ref{subsec:Appendix.interface.function_aliasing}.
\item[GA]
  Shorthand for a \textit{grid array}.
\item[GF]
  Shorthand for a \textit{grid function}.
\item[gmake]
  GNU version of the {\tt make} utility.
\item[ghost zone] 
  A set of points added for parallelisation purposes to a block of a
  grid lying on one processor, corresponding to points on the boundary
  of an adjoining block of the grid lying on another processor.
  Points from the boundary of the one block are copied (via an
  inter-processor communication mechanism) during synchronisation
  to the corresponding ghost zone of the other block, and vice versa.
  In single processor runs there are no ghost zones.
  Contrast with symmetry or boundary zones.
  See Section~\ref{sec:ghost_size}.
\item[grid]
  Short for \textit{computational grid}.
\item[grid array]
  A \textit{grid variable} whose global size need not be that of the
  computational grid; instead, the size is declared explicitly in an
  \verb|interface.ccl| file.  
\item[grid function]
  A \textit{grid variable} whose global size is the size of the
  computational grid.  (See also \textit{local array}.)
  From another perspective,
  \textit{grid functions} are functions (of any of the Cactus
  data types (see Section~\ref{sect-ThornWriting/DataTypes})
  defined on the domain of grid points.
  Typically, grid functions are used to discretely approximate functions
  defined on the domain $\Re^n$, with \textit{finite differencing}
  used to approximate partial derivatives.
\item[grid hierarchy]
  A \textit{computational grid}, and the \textit{grid variables} associated
  with it.
\item[grid point]
  A spatial point in the \textit{computational grid}.
\item[grid scalar]
  A \textit{grid variable} with index zero,
  i.e. just a number on each processor.
\item[grid variable]
  A variable which is passed through the flesh interface, either between 
  thorns or between routines of the same thorn.
  This implies the variable is related to the computational grid, as opposed
  to being an internal variable of the thorn or one of its routines. 
  \textit{grid scalar}, \textit{grid function}, and \textit{grid array}
  are all examples of \textit{grid variables}. 
  See Sections~\ref{sec:cactus_variables-groups},
  \ref{subsec:Appendix.interface-variables}
\item[GNATS]
  The GNU program we use for reporting and tracking bugs, comments and
  suggestions.
\item[GNU]
  \textit{GNU's Not Unix}: a freely-distributable code project.  
  See \url{http://www.gnu.org/}.
\item[GV]
  Shorthand for \textit{grid variable}.
\item[handle]
  A signed integer value $>= 0$ passed by many Cactus routines and
  used to represent a dynamic data or code object.
\item[HDF5]
  \textit{Hierarchical Data Format} version~5, an API, subroutine library, and
  file format for storing structured data.  An HDF5 file can
  store both data (for example, Cactus grid variables), and meta data
  (data describing the other data, for example, Cactus coordinate
  systems).
  See Section~\ref{subsec:cowiexpa}, also
  \url{https://www.hdfgroup.org/HDF5/}.
\item[implementation]
  Defines the interface that a thorn presents to the rest of a Cactus program.
  See Section~\ref{sec:implementations}.
\item[inherit] A thorn that \textit{inherits} from another implementation
  can access all the other implementation's public variables.
  See Section~\ref{sec:Appendix.interface}, \ref{subsec:interface_ccl}.
\item[interface]
\item[interpolation]
  Given a set of grid variables and interpolation points (points in the
  grid coordinate space, which are typically distinct from the grid points),
  interpolation is the act of producing values for the grid variables 
  at each interpolation point over the entire grid hierarchy.  
  (Contrast with \textit{local interpolation}.)
\item[local array]
  An array that is declared in thorn code, but not declared in the thorn's
  \verb|interface.ccl|, as opposed to a \textit{grid array}.
\item[local interpolation]
  Given a set of grid variables and interpolation points (points in the
  grid coordinate space ,which are typically distinct from the grid points),
  interpolation is the act of producing values for the grid variables
  at each interpolation point on a particular grid.  
  (Contrast with \textit{interpolation}.)
\item[Makefile]
  The default input file for \texttt{make} (or \texttt{gmake}).  Includes
  rules for building targets.
\item[make] A system for building software.  It uses rules involving
  dependencies of one part of software on another, and information of what
  has changed since the last build, to determine what parts need to be
  built.
\item[MPI]
  \textit{Message Passing Interface}, an API and software library for sending
  messages between processors in a multiprocessor system.
  See Section~\ref{subsec:cowiexpa}.
\item[multi-patch]
\item[mutual recursion]
  See \textit{recursion, mutual}.
\item[NUL character]
  The C programming language uses a ``NUL character'' to terminate
  character strings.  A NUL character has the integer value zero, but
  it's useful to write it as \verb|'\0'|, to emphasize to human readers
  that this has type \verb|char| rather than \verb|int|.
\item[null pointer, NULL pointer]
  C defines a ``null pointer'', often (slightly incorrectly) called
  a ``NULL pointer'', which is guaranteed not to point to any object.
  You get a null pointer by converting the integer constant 0 to a
  pointer type, e.g.\ \verb|int* ptr = 0;|.%%%
\footnote{%%%
       Note that if you have an expression which
       has the value zero, but which isn't an
       integer constant, converting this to a
       pointer type is \emph{not} guaranteed to
       give a NULL pointer, e.g.:\\
       {\tt int i = 0;}\\
       {\tt int* ptr = i;     /* ptr is NOT guaranteed to be a NULL pointer! */}\\
       }%%%

  Many programmers prefer to use the predefined macro \verb|NULL|
  (defined in \verb|<stdlib.h>|, \verb|<stdio.h>|, and possibly other
  system header files) to create null pointers,
  e.g.\ \verb|int* ptr = NULL;|, to emphasize to human readers that
  this is a null \emph{pointer} rather than ``just'' the integer zero.

  Note that it is explicitly \emph{not} defined whether a null pointer
  is represented by a bit pattern of all zero bits---this varies from
  system to system, and there are real-world systems where null pointers
  are, in fact, \emph{not} represented this way.

  For further information, see the section ``Null pointers''
  in the (excellent) {\tt comp.lang.c FAQ}, available online at
  \url{http://www.eskimo.com/~scs/C-faq/top.html}.
\item[parallelisation]
  The process of utilising multiple computer processors to work on different
  parts of a computational problem at the same time, in order to obtain a
  solution of the problem more quickly.  Cactus achieves parallelisation
  by means of \textit{domain decomposition}.
\item[parameter]
  A variable that controls the run time behaviour of the Cactus executable.
  Parameters have default values which can be set in a
  \textit{parameter file}. (See Chapter~\ref{chap:Cactus_parameters}).
  The flesh has parameters; thorn parameters are made available to the rest
  of Cactus by describing them in the thorn's 
  {\tt param.ccl} file (See Appendix~\ref{sec:Appendix.param}).
\item[parameter file]
  (Also called \textit{par file}.) A text file used as the input of a
  Cactus program, specifying initial values of thorn parameters.
  See Section~\ref{sec:Parameter_File}.
\item[processor topology]
\item[PUGH]
  The default driver thorn for Cactus which uses MPI.
  See Section~\ref{sec:required_software}.
\item[PVM]
  \textit{Parallel Virtual Machine}, provides interprocessor communication.
  See Section~\ref{sec:required_software}.
\item[recursion, mutual]
  See \textit{mutual recursion}.
\item[reduction]
  Given a set of grid variables on a computational grid, \textit{reduction} 
  is the process of producing values for the variables on a proper subset of
  points from the grid.
\item[scheduler]
  The part of the Cactus flesh that determines the order and circumstances
  in which to execute Cactus routines.  Thorn functions and schedule groups 
  are registered with the flesh via the thorn's {\tt schedule.ccl} file to
  be executed in a certain schedule bin, before or after another function 
  or group executes, and so forth.
  See section~\ref{sec:Appendix.schedule}~\ref{chap:scheduling},
\item[schedule bin]
  One of a set of special timebins pre-defined by Cactus.
  See Section \ref{sec:Appendix.schedule_bins} for a list.
\item[schedule group]
  A timebin defined by a thorn, in its {\tt schedule.ccl} file (see
  Appendix \ref{sec:Appendix.schedule}).
  Each schedule group must be defined to occur in a Cactus schedule bin or
  another schedule group.  
  See Chapter~\ref{chap:scheduling}, \ref{scheduling:schedule_bins}.
\item[shares] An implementation may \textit{share} restricted parameters
  with another implementation, which means the other implementation can
  get the parameter values, and if the parameters are steerable, it can
  change them.
  See Section~\ref{sec:Appendix.param} \ref{subsec:param_ccl}.
\item[steerable parameter]
  A parameter which can be changed at any time after the program has been
  initialised.  See Section~\ref{sec:Cactus_parameters.steerable}.
\item[symmetry operation]
  A grid operation that is a manifestation of a geometrical symmetry,
  especially rotation or reflection.
\item[symmetry zone]
  A set of points laying at the edge of the computational grid and
  containing data obtained by some symmetry operation from
  another part of the same grid.
  (Contrast with \textit{boundary zone}, \textit{ghost zone}.)
\item[synchronisation]
  The process of copying information from the outer part of a 
  computational interior on one processor to the corresponding ghost zone 
  (see) on another processor.  Also refers to a special Cactus timebin
  corresponding to the occurrence of this process. 
  See Section~\ref{sec:ghost_size}.
\item[TAGS]
  See Section~\ref{sec:Appendix.tags}.
\item[target]
  A \textit{make target} is the name of a set of rules for 
  \texttt{make} (or \texttt{gmake}).  When the target is included in the
  command line for \texttt{make}, the rules are executed, usually to
  build some software.
\item[test suite]
  See Sections~\ref{sec:testing}, \ref{sec:adding_test_suite}.
\item[thorn]
  A collection of subroutines defining a Cactus interface.  
  See Chapters~\ref{chap:thorn_concepts}, \ref{chap:thorn_anatomy}.
\item[ThornList]
  A file used by the Cactus CST to determine which thorns to compile 
  into a Cactus executable
  (see Section~\ref{sec:gmtafobuanadco}, \ref{sec:cointh}). 
  Can also be used to determine which thorns
  to check out from SVN. (see Section~\ref{sec:checkout}).
  A ThornList for each Cactus configuration lies in the configuration
  subdirectory of the Cactus {\tt configs} directory.
\item[time bin]
  A time interval in the duration of a Cactus run wherein the flesh 
  runs specified routines.  See \textit{scheduler}, \textit{schedule bin}.
\item[time level]
\item[timer]
  A Cactus API for reporting time.  See Section~\ref{sec:timers}.
\item[trigger]
\item[unigrid]
\item[wrapper]

\end{Lentry}

%%%%%%%%%%%%%%%%%%%%%%%%%%%%%%%%%%%%%%%%%%%%%%%%%%%%%%%%%%%%%%%%%%%%%%%%%%%%%%%%
%%%%%%%%%%%%%%%%%%%%%%%%%%%%%%%%%%%%%%%%%%%%%%%%%%%%%%%%%%%%%%%%%%%%%%%%%%%%%%%%
%%%%%%%%%%%%%%%%%%%%%%%%%%%%%%%%%%%%%%%%%%%%%%%%%%%%%%%%%%%%%%%%%%%%%%%%%%%%%%%%

\chapter{Configuration File Syntax}
\label{sec:Appendix.ccl}

\section{General Concepts}

Each thorn is configured by three compulsory and one optional files in the
top level thorn directory:
\begin{itemize}
\item{} {\tt interface.ccl}
\item{} {\tt param.ccl}
\item{} {\tt schedule.ccl}
\item{} {\tt configuration.ccl} (optional)
\end{itemize}
These files are written in the \textit{Cactus Configuration Language} which is
case insensitive.

% A note on optional arguments and square brackets:  If e.g.\ a variable or
% include file is provided by this thorn, then the corresponding
% section of the (in this case) interface.ccl is not optional.  Thus
% these sections are not enclosed in [] as a whole (though parts of them can of
% course be enclosed in []).

\section{interface.ccl}
\label{sec:Appendix.interface}

The interface configuration file consists of:
\begin{itemize}
\item A header block giving details of the thorn's relationship with
other thorns.
\item A block detailing which include files are used from other
thorns, and which include files are provided by this thorn.
\item Blocks detailing aliased functions provided or used by this thorn.
\item A series of blocks listing the thorn's global variables.
\end{itemize}
(For a more extensive discussion of Cactus variables, see Chapter
\ref{chap:cactus_variables}.)

\subsection{Header Block}
The header block has the form:
\begin{alltt}
implements: <\var{implementation}>
inherits: <\var{implementation}>, <\var{implementation}>
friend: <\var{implementation}>, <\var{implementation}>
\end{alltt}
where
\begin{itemize}
\item{} The implementation name must be unique among all thorns, except
        between thorns which have the same public and protected variables and
        global and restricted parameters.
\item{} Inheriting from another implementation makes all that implementation's
        public variables available to your thorn. At least one thorn
        providing any inherited implementation must be present at compile time.
        A thorn cannot inherit from itself. Inheritance is transitive
	(if $A$ inherits from $B$, and $B$ inherits from $C$, then
	$A$ also implicitly inherits from $C$),
	but not commutative.
% FIXME: Jonathan and Thomas aren't sure if this is true!
%	(if $A$ inherits from $B$, then $B$ can't inherit from $A$)
\item{} Being a friend of another implementation makes all that
        implementation's protected variables available to your thorn.
	At least one thorn providing an implementation for each friend
	must be present at compile time.
	A thorn cannot be its own friend.
% FIXME: what does "associative" mean here???
	Friendship is associative,
	commutative
% FIXME: Jonathan and Thomas aren't sure whether B *must* be declared this way
%        or is *implicitly* declared this way
%	(if $A$ is a friend of $B$, then $B$ is also a friend of $A$),
	and transitive (i.e.~if $A$ is a friend of $B$,
	and $B$ is a friend of $C$, then $A$ is implicitly a friend of $C$).
\end{itemize}

\subsection{Include Files}
The include file section has the form:
\begin{alltt}
USES INCLUDE [SOURCE|HEADER]: <\var{file_name}>
INCLUDE[S] [SOURCE|HEADER]: <\var{file_to_include}> in <\var{file_name}>
\end{alltt}
The former is used when a thorn wishes to use an include file from
another thorn.  The latter indicates that this thorn adds the code in
\texttt{<\var{file\_to\_include}>} to the include file \texttt{<\var{file\_name}>}.  If
the include file is described as \verb|SOURCE|, the included code is
only executed if the providing thorn is active.
Both default to \verb|HEADER|.

\subsection{Function Aliasing}
\label{subsec:Appendix.interface.function_aliasing}
If any aliased function is to be used or provided by the thorn, then
the prototype must be declared with the form:
\begin{alltt}
<\var{return_type}> FUNCTION <\var{alias}>(<\var{arg1_type}> <\var{intent1}> [ARRAY] <\var{arg1}>, ...)
\end{alltt}
The \texttt{<\var{return\_type}>} must be either \verb|void|,
\verb|CCTK_INT|, \verb|CCTK_REAL|, \verb|CCTK_COMPLEX|,
\verb|CCTK_POINTER|, or \verb|CCTK_POINTER_TO_CONST|.  The keyword
\verb|SUBROUTINE| is equivalent to \verb|void FUNCTION|. The name of
the aliased function \texttt{<\var{alias}>} must contain at least one
uppercase and one lowercase letter and follow the C standard for
function names.  The type of each argument,
\texttt{<\var{arg*\_type}>}, must be either \verb|CCTK_INT|,
\verb|CCTK_REAL|, \verb|CCTK_COMPLEX|, \verb|CCTK_POINTER|,
\verb|CCTK_POINTER_TO_CONST|, or \verb|STRING|.  All string arguments
must be the last arguments in the list.  The intent of each argument,
\texttt{<\var{intent*}>}, must be either \verb|IN|, \verb|OUT|, or
\verb|INOUT|.  An argument may only be modified if it is declared to
have intent \verb|OUT| or \verb|INOUT|.  If the argument is an array
then the prefix \verb|ARRAY| must also be given.

If the argument \texttt{<\var{arg*}>} is a function pointer, then the argument
itself (which will preceded by the return type) should be
\begin{alltt}
CCTK_FPOINTER <\var{function_arg1}>(<\var{arg1_type}> <\var{intent1}> <\var{arg1}>, ...)
\end{alltt}
Function pointers may not be nested. 

If an aliased function is to be required, then the block
\begin{alltt}
REQUIRES FUNCTION <\var{alias}>
\end{alltt}
is required.

If an aliased function is to be (optionally) used, then the block
\begin{alltt}
USES FUNCTION <\var{alias}>
\end{alltt}
is required.

If a function is provided, then the block
\begin{alltt}
PROVIDES FUNCTION <\var{alias}> WITH <\var{provider}> LANGUAGE <\var{providing_language}>
\end{alltt}
is required. As with the alias name, \texttt{<\var{provider}>} must contain at
least one uppercase and one lowercase letter, and follow the C standard
for function names. Currently, the only supported values of
\texttt{<\var{providing\_language}>} are \verb|C| and \verb|Fortran|.


\subsection{Variable Blocks}
\label{subsec:Appendix.interface-variables}
The thorn's variables are collected into groups. This is not only
for convenience, but for collecting like variables together.
Storage assignment, communication assignment, and ghostzone synchronization
take place for groups only.

The thorn's variables are defined by:

\begin{alltt}
[<\var{access}>:]

<\var{data_type}> <\var{group_name}>[[<\var{number}>]] [TYPE=<\var{group_type}>] [DIM=<\var{dim}>]
[TIMELEVELS=<\var{num}>]
[SIZE=<\var{size in each direction}>] [DISTRIB=<\var{distribution_type}>]
[GHOSTSIZE=<\var{ghostsize}>]
[TAGS=<\var{string}>]  ["<\var{group_description}>"]
[\{
 [ <\var{variable_name}>[,]<\var{variable_name}>
   <\var{variable_name}> ]
\} ["<\var{group_description}>"] ]
\end{alltt}
%
(The options {\t TYPE}, {\t DIM}, etc.,\ following {\t <\var{group\_name}>}
must all appear on one line.)  Note that the beginning brace (\{) must
sit on a line by itself; the ending brace (\}) must be preceded by a
carriage return.
%
\begin{itemize}
\item{} {\t \var{access}} defines which thorns can use the following
        groups of variables. {\t \var{access}} can be either
        {\t public}, {\t protected} or {\t private}.
\item{} {\t \var{data\_type}} defines the data type of the variables in the
group.  Supported data types are {\t CHAR}, {\t BYTE}, {\t INT}, {\t REAL}, and
{\t COMPLEX}. 
\item{} {\t \var{group\_name}} must be an alphanumeric name (which may also
contain underscores) which is unique across group and variable names
within the scope of the thorn. A group name is compulsory.
\item{} {\t [\var{number}]}, if present, indicates that this is a
  \emph{vector} group.  The number can be any valid arithmetical
  expression consisting of integers or integer-valued parameters.
  Each variable in that group appears as a one-dimensional array of
  grid variables.  When the variable is accessed in the code, then the
  last index is the member-index, and any other indices are the normal
  spatial indices for a group of this type and dimension.
\item{} {\t TYPE} designates the kind of variables held by the group.
The choices are {\t GF}, {\t ARRAY} or {\t SCALAR}. This field is
optional, with the default variable type being {\t SCALAR}.
\item{} {\t DIM} defines the spatial dimension of the {\t ARRAY} or
{\t GF}.  The default value is {\t DIM=3}.
\item{} {\t TIMELEVELS} defines the number of timelevels a group has if
        the group is of type {\t ARRAY} or {\t GF}, and can take any positive
        value. The default is one timelevel.
\item{} {\t SIZE} defines the number grid-points an {\tt ARRAY} has in each direction.
        This should be a comma-separated list of valid arithmetical
        expressions consisting of integers or integer-valued parameters.
\item{} {\t DISTRIB} defines the processor decomposition of an {\tt ARRAY}.
        {\tt DISTRIB=DEFAULT} distributes {\tt SIZE} grid-points
        across all processors. {\tt DISTRIB=CONSTANT} implies that
        {\tt SIZE} grid-points should be allocated on each
        processor. The default value is {\tt DISTRIB=DEFAULT}.
\item{} {\t GHOSTSIZE} defines number of ghost zones in each dimension
of an {\tt ARRAY}.  It defaults to zero.
\item{} {\t TAGS} defines an optional string which is used to create a
	set of key-value pairs associated with the group. The keys are case
	independent.  The string (which must be deliminated by single or
	double quotes) is interpreted by the function
	{\t Util\_TableSetFromString()}, which is described in the
        Reference Manual.\\
        Currently the CST parser and the flesh do not evaluate any information
        passed in an optional {\t TAGS} string. Thorns may do so by
        querying the key/value table information for a group by using
        {\t CCTK\_GroupTagsTable()} and the
        appropriate {\t Util\_TableGet*()} utility functions
        (see the ReferenceManual for detailed descriptions).\\
        For a list of currently supported {\t TAGS} key-value table information,
        please refer to the corresponding chapter in the documentation of the
        \verb|CactusDoc| arrangement. (Section \ref{sec:OtherGmakeTargetsDoc} on
        page \pageref{sec:OtherGmakeTargetsDoc} explains how to build this
        documentation).
\item{} The (optional) block following the group declaration line,
contains a list of variables contained in the group. All variables in
a group have the same data type, variable type, dimension and
distribution. The list
can be separated by spaces, commas, or new lines. The variable names
must be unique within the scope of the thorn.  A variable can only be
a member of one group. The block must be delimited by brackets on new
lines.  If no block is given after a group declaration line, a
variable with the same name as the group is created. Apart from this case, 
a group name cannot be the same as the name of any variable seen by this thorn.
\item{} An optional description of the group can be given on the last
line.  If the variable block is omitted, this description can be given
at the end of the declaration line.
\end{itemize}

The process of sharing code among thorns using include files is
discussed in Section~\ref{sec:includefiles}.

\section{param.ccl}
\label{sec:Appendix.param}

The parameter configuration file consists of a list of
\textit{parameter object specification items} (OSIs) giving the type and
range of the parameter separated by optional
\textit{parameter data scoping items} (DSIs), which detail access to the
parameter.  (For a more extensive discussion of Cactus parameters, see Chapter
\ref{chap:Cactus_parameters}.)

\subsection{Parameter Data Scoping Items}

\begin{alltt}
<\var{access}>:
\end{alltt}

The keyword {\t \var{access}} designates that all parameter object specification
items, up to the next parameter data scoping item, are in the same
protection or scoping class. {\tt \var{access}} can take the values:
\begin{Lentry}
\item[{\tt global}] all thorns have access to global parameters
\item[{\tt restricted}] other thorns can have access to these
                           parameters, if they specifically request
                           it in their own param.ccl
\item[{\tt private}] only your thorn has access to private parameters
\item[{\tt shares}] in this case, an {\t implementation} name must
follow the colon. It declares that all the parameters in the following
scoping block are restricted variables from the specified {\tt
implementation}.  (Note: only one implementation can be specified
on this line.)
\end{Lentry}


\subsection{Parameter Object Specification Items}
\label{subsec:Appendix.param.specification_items}

\begin{alltt}
[EXTENDS|USES] <\var{parameter type}> <\var{parameter name}>[[<\var{len}>]] "<\var{parameter description}>" 
[AS <\var{alias}>] [STEERABLE=<NEVER|ALWAYS|RECOVER>] 
[ACCUMULATOR=<\var{expression}>] [ACCUMULATOR-BASE=<\var{parameter name}>]
\{
  <\var{parameter values}>
\} <\var{default value}>
\end{alltt}
where the options {\t AS}, {\t STEERABLE}, etc., following {\t <\var{parameter description}>},
must all appear in one line.
  Note that the beginning brace ({\t\{}) must
sit on a line by itself; the ending brace ({\t\}}) must be at the
beginning of a line followed by \var{<default value>} on that same line.

\begin{itemize}
\item
  The \var{parameter values} depend on the \var{parameter type}, which may
  be one of the following:
  \begin{Lentry}
  \item[{\t INT}]
    The specification of \texttt{\var{parameter value}}s takes the form of one
    or more lines, each of the form
    \begin{alltt}
      \var{<range description>} [::"\var{<comment describing this range>}"]
    \end{alltt}
    Here, a \var{<range description>} specifies a set of integers,
    and has one of the following forms:
    \begin{alltt}
      *                           \# means any integer
      \var{<integer>}                   \# means only \var{<integer>}
      \var{<lower bound>}:\var{<upper bound>} \# means all integers in the range
                                  \# from \var{<lower bound>} to \var{<upper bound>}
      \var{<lower bound>}:\var{<upper bound>}:\var{<positive step>}
                                  \# means all integers in the range
                                  \# from \var{<lower bound>} to \var{<upper bound>}
                                  \# in steps of \var{<positive step>}
    \end{alltt}
    where \var{<lower bound>} has one of the forms
    \begin{alltt}
      \var{<empty field>}   \# means no lower limit
      *               \# means no lower limit
      \var{<integer>}       \# means a closed interval starting at \var{<integer>}
      [\var{<integer>}      \# also means a closed interval starting at \var{<integer>}
      (\var{<integer>}      \# means an open interval starting at \var{<integer>}
    \end{alltt}
    and \var{<upper bound>} has one of the forms
    \begin{alltt}
      \var{<empty field>}   \# means no upper limit
      *               \# means no upper limit
      \var{<integer>}       \# means a closed interval ending at \var{<integer>}
      \var{<integer>}]      \# also means a closed interval ending at \var{<integer>}
      \var{<integer>})      \# means an open interval ending at \var{<integer>}
    \end{alltt}

  \item[{\t REAL}]
    The range specification is the same as with integers, except that here,
    no \var{step} implies a continuum of values.  Note that numeric
    constants should be expressed as in C (e.g.\ {\t 1e-10}).  Note
    also that one cannot use the Cactus types such as {\t CCTK\_REAL4}
    to specify the precision of the parameter; parameters always have
    the default precision.

  \item[{\t KEYWORD}]
    Each entry in the list of acceptable values for a keyword has the
    form
    \begin{alltt}
      <\var{keyword value}>, <\var{keyword value}> :: "<\var{description}>"
    \end{alltt}
    Keyword values should be enclosed in double quotes.  The double
    quotes are mandatory if the keyword contains spaces.

  \item[{\t STRING}]
    Allowed values for strings should be specified using regular
    expressions. To allow any string, the regular expression {\tt ""}
    should be used.  (An empty regular expression matches anything.)
    Regular expressions and string values should be enclosed in double
    quotes.  The double quotes are mandatory if the regular expression
    or the string value is empty or contains spaces.

  \item[{\t BOOLEAN}]
  No \texttt{\var{parameter values}} should be specified for a boolean
    parameter. The default value for a boolean can be
    \begin{itemize}
    \item True: {\t 1}, {\t yes}, {\t y}, {\t t}, {\t true}
    \item False: {\t 0}, {\t no}, {\t n}, {\t f}, {\t false}
    \end{itemize}
    Boolean values may optionally be enclosed in double quotes.

  \end{Lentry}

\item
  The \var{parameter name} must be unique within the scope of the
  thorn.

\item
  The \var{default value} must match one of the ranges given in the
  \var{parameter type}

\item
  A thorn can declare that it {\t EXTENDS} a parameter of another
  thorn. This allows it to declare additional acceptable values.  By
  default, it is acceptable for two thorns to declare the same value as
  acceptable.

\item
  If the thorn wants to simply use a parameter from another thorn,
  without declaring additional values, use {\t USES} instead.

\item
{\tt [\var{len}]} (where {\tt \var{len}} is an integer), if present,
  indicates that this is an \textit{array} parameter of {\tt \var{len}}
  values of the specified type. (Note that the notation used above for the
  parameter specification breaks down here, as there must be square brackets
  around the length).

\item
  \var{alias} allows a parameter to appear under a different name in this
  thorn, other than its original name in another thorn.  The name, as seen in
  the parameter file, is unchanged.

\item
  {\t STEERABLE} specifies when a parameter value may be changed.  By
  default, parameters may not be changed after the parameter file has
  been read, or on restarting from checkpoint.  This option relaxes
  this restriction, specifying that the parameter may be changed at
  recovery time from a parameter file or at any time using the flesh
  routine {\tt CCTK\_ParameterSet}---see the Reference Guide.

  The value {\tt RECOVERY} is used in checkpoint/recovery situations,
  and indicates that the parameter may be altered until the value is
  read in from a recovery par file, but not after.

\item
  {\t ACCUMULATOR} specifies that this is an \textit{accumulator}
  parameter.  Such parameters cannot be set directly, but are set by
  other parameters who specify this one as an {\tt ACCUMULATOR-BASE}.
  The expression is a two-parameter arithmetical expression of $x$ and
  $y$.  Setting the parameter consists of evaluating this expression
  successively, with $x$ being the current value of the parameter (at
  the first iteration this is the default value), and $y$ the value of
  the setting parameter.  This procedure is repeated, starting from
  the default value of the parameter, each time one of the setting
  parameters changes.

\item
  {\t ACCUMULATOR-BASE} specifies the name of an {\tt ACCUMULATOR}
  parameter which this parameter sets.

\end{itemize}

\section{schedule.ccl}
\label{sec:Appendix.schedule}

(A more extensive discussion of Cactus scheduling is provided in Chapter
\ref{chap:scheduling}.)
A schedule configuration file consists of:
\begin{itemize}

\item{} \textit{Assignment statements} to switch on storage for
  grid variables for the entire duration of program execution.

\item{} \textit{Schedule blocks} to schedule a subroutine from a thorn
  to be called at specific times during program execution in a given manner.

\item {} \textit{Conditional statements} for both assignment statements and
  schedule blocks to allow them to be processed depending on parameter values.

\end{itemize}

\subsection{Assignment Statements}

\textit{Assignment statements}, currently only assign storage.

These lines have the form:
\begin{alltt}
[STORAGE: <\var{group}>[\var{timelevels}], <\var{group}>[\var{timelevels}]]
\end{alltt}

If the thorn is active, storage will be allocated, for the given groups,
for the duration of program execution (unless storage is explicitly
switched off by some call to {\tt CCTK\_DisableGroupStorage} within a
thorn). 

The storage line includes the number of timelevels to activate storage
for, this number can be from 1 up to the maximum number or timelevels
for the group, as specified in the defining {\tt interface.ccl}
file. If the maximum number of timelevels is 1 (the default), this
number may be omitted. Alternatively \var{timelevels} can be the name of a
parameter accessible to the thorn. The parameter name is the same as used in C
routines of the thorn, fully qualified parameter names of the form
\texttt{\var{thorn}::\var{parameter}} are not allowed. In this case 0 (zero)
\var{timelevels} can be requested, which is equivalent to the {\tt STORAGE}
statement being absent.

The behaviour of an assignment statement is independent of its
position in the schedule file (so long as it is outside a schedule
block). 

\subsection{Schedule Blocks}

Each \textit{schedule block} in the file {\t schedule.ccl} must have the syntax

\begin{alltt}
schedule [GROUP] <\var{function name}|\var{group name}> AT|IN <\var{time}> \verb|\|
     [AS <\var{alias}>] \verb|\|
     [WHILE <\var{variable}>] [IF <\var{variable}>] \verb|\|
     [BEFORE|AFTER <\var{function name}>|(<\var{function name}> <\var{function name}> ...)] \verb|\|
\{
  [LANG: <\var{language}>]
  [OPTIONS:       <\var{option}>,<\var{option}>...]
  [TAGS:          <\var{keyword=value}>,<\var{keyword=value}>...]
  [STORAGE:       <\var{group}>[\var{timelevels}],<\var{group}>[\var{timelevels}]...]
  [READS:         <\var{group}>,<\var{group}>...]
  [WRITES:        <\var{group}>,<\var{group}>...]
  [TRIGGER:       <\var{group}>,<\var{group}>...]
  [SYNCHRONISE:   <\var{group}>,<\var{group}>...]
  [OPTIONS:       <\var{option}>,<\var{option}>...]
\} "\var{Description of function}"
\end{alltt}

\begin{Lentry}
  \item[{\tt GROUP}] Schedule a schedule group with the same options
  as a schedule function.  The schedule group will be created if it doesn't exist.

  \item[{\tt <\var{function name}|\var{group name}>}] The name of a function or a
  schedule group to be scheduled.  Function and schedule group names
  are case sensitive.

  \item[{\tt <\var{group}>}] A group of grid variables. Variable groups
  inherited from other thorns may be used, but they must then be fully
  qualified with the implementation name.

  \item[{\tt AT}] Functions can be scheduled to run at the Cactus
  schedule bins, for example, {\tt CCTK\_EVOL}, and {\tt CCTK\_STARTUP}. A
  complete list and description of these is provided in
  Appendix~\ref{sec:Appendix.schedule_bins}.  The initial letters
  {\tt CCTK\_} are optional. Grid variables cannot be used in the
  {\tt CCTK\_STARTUP} and {\tt CCTK\_SHUTDOWN} timebins.

  \item[{\tt IN}] Schedules a function or schedule group to run in a
  schedule group, rather than in a Cactus timebin.

  \item[{\tt AS}] Provides an alias for a function or schedule group
  which should be used for scheduling before, after or in.  This can
  be used to provide thorn independence for other thorns scheduling
  functions, or schedule groups relative to this one.

  \item[{\tt WHILE}] Executes a function or schedule group until the given
    variable (which must be a fully qualified integer grid scalar) has
    the value zero.

  \item[{\tt IF}] Executes a function or schedule group only if the given
    variable (which must be a fully qualified integer grid scalar) has
    a non-zero value.

  \item[{\tt BEFORE/AFTER}] Takes a function name, a function alias,
  a schedule group name, or a parentheses-enclosed whitespace-separated
  list of these.  (Any names that are not provided by an active thorn
  are ignored.)  Note that a single schedule block may have multiple
  {\tt BEFORE/AFTER} clauses.
  See Section~\ref{chap:scheduling}
  (``Scheduling'') in the Cactus Users' Guide for more information
  about {\tt BEFORE/AFTER} clauses.

  \item[{\tt LANG}] The code language for the function (either {\tt C} or {\tt
    FORTRAN}). No language should be specified for a schedule group.

  \item[\texttt{OPTIONS}] Schedule options are used for mesh
    refinement and multi-block simulations, and they determine
    ``where'' a routine executes. Possible options are:
    \begin{description}
    \item[\texttt{meta}]
    \item[\texttt{meta\_early}]
    \item[\texttt{meta\_late}]
    \item[\texttt{global}]
    \item[\texttt{global\_early}]
    \item[\texttt{global\_late}]
    \item[\texttt{level}]
    \item[\texttt{singlemap}]
    \item[\texttt{local}] (default, may be omitted)
    \end{description}
    (Only one of these options may be used.)
    These options can be combined with the following:
    \begin{description}
    \item[\texttt{loop\_meta}]
    \item[\texttt{loop\_global}]
    \item[\texttt{loop\_level}]
    \item[\texttt{loop\_singlemap}]
    \item[\texttt{loop\_local}]
    \end{description}
    (At most one of the \texttt{loop\_...} options may be used.)

  \item[\texttt{TAGS}] Schedule tags. These tags must have the form
    \texttt{keyword=value}, and must be in a syntax accepted by
    \texttt{Util\_TableCreateFromString}.

  \item[{\tt STORAGE}] List of variable groups which should have storage
  switched on for the duration of the function or schedule group.
  Each group must specify how many timelevels to activate storage for,
  from 1 up to the maximum number for the group as specified in the
  defining {\tt interface.ccl} file. If the maximum is 1 (the default)
  this number may be omitted. Alternatively \var{timelevels} can be the
  name of a parameter accessible to the thorn. The parameter name is the
  same as used in C routines of the thorn, fully qualified parameter names
  of the form \texttt{\var{thorn}::\var{parameter}} are not allowed. In this
  case 0 (zero) \var{timelevels} can be requested, which is equivalent to
  the {\tt STORAGE} statement being absent.

\item[\texttt{READS}] \texttt{READS} is used to declare which grid
  variables are read by the routine. This information is used e.g.\ to
  determine which variables need to be copied between host and device
  for OpenCL or CUDA kernel. This information can also be used to
  ensure that all variables that are read have previously been written
  by another routine.

\item[\texttt{WRITES}] \texttt{WRITES} is used to declare which grid
  variables are written by the routine. This information is used e.g.\ to
  determine which variables need to be copied between host and device
  for OpenCL or CUDA kernel. This information can also be used to
  ensure that all variables that are read have previously been written
  by another routine.

  \item[{\tt TRIGGER}] List of grid variables or groups to be used as
  triggers for causing an {\tt ANALYSIS} function or group to be
  executed.  Any schedule block for an analysis function or analysis
  group may contain a {\tt TRIGGER} line.

  \item[{\tt SYNCHRONISE}] List of groups to be synchronised, as soon
    as the function or schedule group is exited.

  \item[{\tt OPTIONS}] List of additional options (see below) for the scheduled function or group of functions

\end{Lentry}

\subsubsection{Allowed Options}

\label{app:allopts}

Cactus understands the following options.  These options are
interpreted by the driver, not by Cactus.  The current set of options
is useful for Berger-Oliger mesh refinement which has subcycling in
time, and for multi-patch simulations in which the domain is split
into several distinct patches.  Given this, the meanings of the
options below is only tentative, and their exact meaning needs to be
obtained from the driver documentation.  The standard driver PUGH
ignores all options.

Option names are case-insensitive.  There can be several options given
at the same time.

\begin{Lentry}
  
\item[{\tt META}] This routine will only be called once, even if
  several simulations are performed at the same time.  This can be
  used, for example, to initialise external libraries, or to set up
  data structures that live in global variables.
  
\item[{\tt META-EARLY}] This option is identical to to {\tt META}
  option with the exception that the routine will be called together
  with the routines on the first subgrid.
  
\item[{\tt META-LATE}] This option is identical to to {\tt META}
  option with the exception that the routine will be called together
  with the routines on the last subgrid.
  
\item[{\tt GLOBAL}] This routine will only be called once on a grid
  hierarchy, not for all subgrids making up the hierarchy.  This can
  be used, for example, for analysis routines which use global
  reduction or interpolation routines, rather than the local subgrid
  passed to them, and hence should only be called once.
  
\item[{\tt GLOBAL-EARLY}] This option is identical to to {\tt GLOBAL}
  option with the exception that the routine will be called together
  with the routines on the first subgrid.
  
\item[{\tt GLOBAL-LATE}] This option is identical to to {\tt GLOBAL}
  option with the exception that the routine will be called together
  with the routines on the last subgrid.
  
\item[{\tt LEVEL}] This routine will only be called once on any
  ``level'' of the grid hierarchy.  That is, it will only be called
  once for any set of sub-grids which have the same
  \texttt{cctk\_levfac} numbers.
  
\item[{\tt SINGLEMAP}] This routine will only be called once on any of
  the ``patches'' that form a ``level'' of the grid hierarchy.
  
\item[{\tt LOCAL} (this is the default)] This routine will be called
  on every ``component''.

\end{Lentry}

When the above options are used, it is often the case that a certain
routine should, e.g.\ be called at the time for a \texttt{GLOBAL}
routine, but should actually loop over all ``components''.  The
following set of options allows this:

\begin{Lentry}
  
\item[{\tt LOOP-META}] Loop once.
  
\item[{\tt LOOP-GLOBAL}] Loop over all simulations.
  
\item[{\tt LOOP-LEVEL}] Loop over all ``levels''.
  
\item[{\tt LOOP-SINGLEMAP}] Loop over all ``patches''.
  
\item[{\tt LOOP-LOCAL}] Loop over all ``components''.

\end{Lentry}

For example, the specification
\begin{alltt}
  OPTIONS: global loop-local
\end{alltt}
schedules a routine at the time when a \texttt{GLOBAL} routine is
scheduled, and then calls the routine in a loop over all
``components''.


\subsection{Conditional Statements}

Any schedule block or assignment statements can be optionally
surrounded by conditional {\t if-elseif-else}
constructs using the parameter data base. These can be nested,
and have the general form:

\begin{alltt}
if (<\var{conditional-expression}>)
\{
  [<\var{assignments}>]
  [<\var{schedule blocks}>]
\}\end{alltt}

<\var{conditional-expression}> can be any valid C construct evaluating
to a truth value.
Such conditionals are evaluated only at program startup, and are used
to pick between different static schedule options.  For dynamic
scheduling, the {\tt SCHEDULE WHILE} construction should be used.

Conditional constructs cannot be used inside a schedule block.

%%%%%%%%%%%%%%%%%%%%%%%%%%%%%%%%%%%%%%%%%%%%%%%%%%%%%%%%%%%%%%%%%%%%%%%%%%%%%%%%

\section{configuration.ccl}
\label{sec:Appendix.configuration.ccl}

[{\bf NOTE:} The configuration.ccl is still relatively new, and not
all features listed below may be fully implemented or functional.]

A configuration.ccl file defines {\bf capabilities} which a thorn
either provides or requires, or may use if available.  Unlike {\bf
implementations}, only one thorn providing a particular capability may
be compiled into a configuration at one time.  Thus, this mechanism may
be used to, for example: provide access to external libraries; provide
access to functions which other thorns must call, but are too complex
for function aliasing; or to split a thorn into several thorns, all of
which require some common (not aliased) functions.

A configuration options file can contain any number of the following
sections:

\begin{itemize}

\item

\begin{alltt}
PROVIDES <\var{Capability}>
\{
  SCRIPT <\var{Configuration script}>
  [VERSION <\var{Version String}>]
  LANG <\var{Language}>
  [OPTIONS [<option>[,<option>]...]]
\}
\end{alltt}

Informs the CST that this thorn provides a given capability, and that
this capability has a given detection script which may be used to
configure it (e.g. running an autoconf script or detecting an external
library's location).  The script should output configuration
information on its standard output---the syntax is described below
in Section \ref{sec:Appendix.configuration.ccl.configscript}.  The
script may also indicate the failure to detect a capability by
returning a non-zero exit code;  this will stop the build after the
CST stage.

All capabilities have an optional version attached to them, so that other
thorns can depend on such a specific version. Version strings can only contain
letters, numbers, or ``.+-:'' (dot, plus, minus, colon), and have to start with
a number. Specifying a version number is optional. If none is given, ``0.0.1''
is assumed.

Scripts can be in any language.  If an interpreter is needed to run
the script, for example \verb|Perl|, this should be indicated by the
\verb|LANG| option.

The specified options are checked for in the original configuration,
and any options passed on the command line (including an `options'
file) at compile time when the thorn is added, or if the CST is
rerun.  These options need be set only once, and will be remembered
between builds.

\item

\begin{alltt}
REQUIRES  <\var{Capability}> [(\var{Comparison operator} \var{Version string})]
\end{alltt}

Informs the CST that this thorn requires a certain capability to be
present.  If no thorn providing the capability is in the ThornList,
the build will stop after the CST stage.

Optionally, thorns can depend on a capability version. This has to be enclosed in
parentheses, following the capability name. Inside the parenthesis, first a comparison
operator determines the kind of dependency on the requested capability version, which
has to be attached without spaces. Valid operators and their meaning are:
\begin{itemize}
 \item[$<<$] Requesting capability strictly older than given version
 \item[$<=$] Requesting capability older or equal to given version
 \item[$=$]  Requesting capability exactly equal to given version
 \item[$>=$] Requesting capability newer or equal to given version
 \item[$>>$] Requesting capability strictly newer than given version.
\end{itemize}
Version strings are compared from left to right. First the initial part of each
string consisting entirely of digits is determined. The integer values of
these two parts are compared. If a difference is found it is returned. Then the
initial part of the remainder of each string which consists entirely of
non-digit characters is determined. The two parts are compared lexically, and
any difference found is returned as the result of the comparison. The lexical
comparison is a comparison of ASCII values. These two steps (comparing and
removing initial non-digit strings and initial digit strings) are repeated
until a difference is found or both strings are exhausted. Spaces inside the
parentheses are allowed, except inside the operator or version string.

Example:

\begin{alltt}
REQUIRES ExampleCapability (>=2.43.2dev-1)
\end{alltt}


\item

\begin{alltt}
OPTIONAL <\var{Capability}>
\{
  DEFINE <\var{macro}>
\}
\end{alltt}

Informs the CST that this thorn may use a certain capability, if a
thorn providing it is in the ThornList.  If present, the preprocessor
macro, \verb|macro|, will be defined and given the value ``1''.

\end{itemize}

\subsection{Configuration Scripts}
\label{sec:Appendix.configuration.ccl.configscript}

The configuration script may tell the CST to add certain features to
the Cactus environment---either to the make system or to header
files included by thorns.  It does this by outputting lines to its
standard output:

\begin{itemize}

% [[How about the MESSAGE field?]]

\item

\begin{alltt}
BEGIN DEFINE 
<text>
END DEFINE
\end{alltt}

Places a set of definitions in a header file which will be included by
all thorns using this capability (either through an OPTIONAL or
REQUIRES entry in their configuration.ccl files).

\item

\begin{alltt}
INCLUDE_DIRECTORY  <directory>
\end{alltt}

Adds a directory to the include path used for compiling files in
thorns using this capability.

\item

\begin{alltt}
BEGIN MAKE_DEFINITION 
<text>
END MAKE_DEFINITION
\end{alltt}

Adds a makefile definition into the compilation of all thorns using
this capability.

\item

\begin{alltt}
BEGIN MAKE_DEPENDENCY 
<text>
END MAKE_DEPENDENCY 
\end{alltt}

Adds makefile dependency information into the compilation of all
thorns using this capability.

\item

\begin{alltt}
LIBRARY <library>
\end{alltt}

Adds a library to the final cactus link.

\item

\begin{alltt}
LIBRARY_DIRECTORY <library>
\end{alltt}

Adds a directory to the list of directories searched for libraries at
link time.

\end{itemize}

No other lines should be output by the script.

%%%%%%%%%%%%%%%%%%%%%%%%%%%%%%%%%%%%%%%%%%%%%%%%%%%%%%%%%%%%%%%%%%%%%%%%%%%%%%%%
%%%%%%%%%%%%%%%%%%%%%%%%%%%%%%%%%%%%%%%%%%%%%%%%%%%%%%%%%%%%%%%%%%%%%%%%%%%%%%%%
%%%%%%%%%%%%%%%%%%%%%%%%%%%%%%%%%%%%%%%%%%%%%%%%%%%%%%%%%%%%%%%%%%%%%%%%%%%%%%%%

\chapter{Utility Routines}

\section{Introduction}

As well as the high-level \verb|CCTK_|* routines, Cactus also
provides a set of lower-level \verb|Util_|* utility routines, which
are mostly independent of the rest of Cactus.  This chapter gives a
general overview of programming with these utility routines.

%%%%%%%%%%%%%%%%%%%%%%%%%%%%%%%%%%%%%%%%%%%%%%%%%%%%%%%%%%%%%%%%%%%%%%%%%%%%%%%%

\section{Key/Value Tables}

%%%%%%%%%%%%%%%%%%%%%%%%%%%%%%%%%%%%%%%%

\subsection{Motivation}

Cactus functions may need to pass information through a generic
interface.  In the past, we have used various ad hoc means to do this,
and we often had trouble passing "extra" information that wasn't
anticipated in the original design.  For example, for periodic output
of grid variables,
\verb|CCTK_OutputVarAsByMethod()| requires that
any parameters (such as hyperslabbing parameters) be appended as an option
string to the variable's character string name.  Similarly, elliptic
solvers often need to pass various parameters, but we haven't had a
good way to do this.

Key/value tables (\textit{tables} for short) provide a clean solution
to these problems.  They're implemented by the \verb|Util_Table|*
functions (described in detail in the Reference Manual).

%%%%%%%%%%%%%%%%%%%%%%%%%%%%%%%%%%%%%%%%

\subsection{The Basic Idea}

Basically, a table is an object which maps strings to almost arbitrary
user-defined data.  (If you know Perl, a table is very much like a
Perl hash table.  Alternatively, if you know Unix shells, a table is
like the set of all environment variables.  As yet another analogy,
if you know Awk, a table is like an Awk associative array.)%%%
\footnote{%%%
	 However, the present Cactus tables implementation
	 is optimized for a relatively small number of
	 distinct keys in any one table.  It will still
	 work OK for huge numbers of keys, but it will be
	 slow.
	 }%%%

More formally, a table is an object which stores a set of \textit{keys}
and a corresponding set of \textit{values}.  We refer to a (key,value)
pair as a table \textit{entry}.

Keys are C-style null-terminated character strings, with the slash
character `{\tt /}' reserved for future expansion.%%%
\footnote{%%%
	 Think of hierarchical tables for storing
	 tree-like data structures.%%%
	 }%%%

Values are 1-dimensional arrays of any of the usual Cactus data types,
described in Section~\ref{sect-ThornWriting/DataTypes}.
A string can be stored by treating it as a 1-dimensional array of
\verb|CCTK_CHAR| (there's an example of this below).

The basic ``life cycle'' of a table looks like this:
\begin{enumerate}
\item	Some code creates it with \verb|Util_TableCreate()|
	or \verb|Util_TableClone()|.
\item	Some code (often the same piece of code, but maybe some
	other piece) sets entries in it using one or more of
	the \verb|Util_TableSet*()|, \verb|Util_TableSet*Array()|,
	\verb|Util_TableSetGeneric()|, \verb|Util_TableSetGenericArray()|,
	and/or \verb|Util_TableSetString()| functions.
\item	Some other piece or pieces of code can get (copies of)
	the values which were set, using one or more of the
	\verb|Util_TableGet*()|, \verb|Util_TableGet*Array()|,
	\verb|Util_TableGetGeneric()|, \verb|Util_TableGetGenericArray()|,
	and/or \verb|Util_TableGetString()| functions.
\item	When everyone is through with a table, some (single)
	piece of code should destroy it with \verb|Util_TableDestroy()|.
\end{enumerate}

There are also convenience functions \verb|Util_TableSetFromString()|
to set entries in a table based on a parameter-file-style string,
and \verb|Util_TableCreateFromString()| to create a table and then
set entries in it based on a parameter-file-style string.

As well, there are ``table iterator'' functions \verb|Util_TableIt*()|
to allow manipulation of a table even if you don't know its keys.

A table has an integer ``flags word'' which may be used to specify
various options, via bit flags defined in \verb|util_Table.h|.
For example, the flags word can be used to control whether keys
should be compared as case sensitive or case insensitive strings.
See the detailed function description of \verb|Util_TableCreate()|
in the Reference Manual for a list
of the possible bit flags and their semantics.

%%%%%%%%%%%%%%%%%%%%%%%%%%%%%%%%%%%%%%%%

\subsection{A Simple Example}
\label{Tables_Simple_Example}

Here's a simple example (in C)%%%
\footnote{%%%
	 All (or almost all) of the table routines
	 are also usable from Fortran.  See the full
	 descriptions in the Reference Manual
	 for details.
	 }%%%
{} of how to use a table:
\begin{verbatim}
#include "util_Table.h"
#include "cctk.h"

/* create a table and set some entries in it */
int handle = Util_TableCreate(UTIL_TABLE_FLAGS_DEFAULT);
Util_TableSetInt(handle, 2, "two");
Util_TableSetReal(handle, 3.14, "pi");

...

/* get the values from the table */
CCTK_INT two_value;
CCTK_REAL pi_value;
Util_TableGetInt(handle, &two_value, "two");    /* sets two_value = 2 */
Util_TableGetReal(handle, &pi_value, "pi");     /* sets pi_value = 3.14 */
\end{verbatim}

Actually, you shouldn't write code like this---in the real world
errors sometimes happen, and it's much better to catch them close to
their point of occurrence, rather than silently produce garbage results
or crash your program.  So, the \emph{right} thing to do is to always
check for errors.  To allow this, all the table routines return a status,
which is zero or positive for a successful return, but negative if
and only if some sort of error has occurred.%%%
\footnote{%%%
	 Often (as in the examples here) you don't care
	 about the details of which error occurred.  But if
	 you do, there are various error codes defined in
	 {\t util\_Table.h} and {\t util\_ErrorCodes.h};
	 the detailed function descriptions in
	 the Reference Manual
	 say which error codes each function can return.
	 }%%%
{}  So, the above example should be rewritten like this:

\begin{verbatim}
#include "util_Table.h"

/* create a table and set some entries in it */
int handle = Util_TableCreate(UTIL_TABLE_FLAGS_DEFAULT);
if (handle < 0)
        CCTK_WARN(CCTK_WARN_ABORT, "couldn't create table!");

/* try to set some table entries */
if (Util_TableSetInt(handle, 2, "two") < 0)
        CCTK_WARN(CCTK_WARN_ABORT, "couldn't set integer value in table!");
if (Util_TableSetReal(handle, 3.14, "pi") < 0)
        CCTK_WARN(CCTK_WARN_ABORT, "couldn't set real value in table!");

...

/* try to get the values from the table */
CCTK_INT two_value;
CCTK_REAL pi_value;
if (Util_TableGetInt(handle, &two_value, "two") < 0)
        CCTK_WARN(CCTK_WARN_ABORT, "couldn't get integer value from table!");
if (Util_TableGetReal(handle, &pi_value, "pi") < 0)
        CCTK_WARN(CCTK_WARN_ABORT, "couldn't get integer value from table!");

/* if we get to here, then two_value = 2 and pi_value = 3.14 */
\end{verbatim}

%%%%%%%%%%%%%%%%%%%%%%%%%%%%%%%%%%%%%%%%

\subsection{Arrays as Table Values}

As well as a single numbers (or characters or pointers), tables can
also store 1-dimensional arrays of numbers (or characters or pointers).%%%
\footnote{%%%
	 Note that the table makes (stores) a \emph{copy} of the array
	 you pass in, so it's somewhat inefficient to store a large array
	 (e.g.~a grid function) this way.  If this is a problem, consider
	 storing a \texttt{CCTK\_POINTER} (pointing to the array) in the table
	 instead.  (Of course, this requires that you ensure that the array still exists whenever that \texttt{CCTK\_POINTER} is used.)
	 }%%%

For example (continuing the previous example):
\begin{verbatim}
static const CCTK_INT a[3] = { 42, 69, 105 };
if (Util_TableSetIntArray(handle, 3, a, "my array") < 0)
        CCTK_WARN(CCTK_WARN_ABORT, "couldn't set integer array value in table!");

...

CCTK_INT blah[10];
int count = Util_TableGetIntArray(handle, 10, blah, "my array");
if (count < 0)
        CCTK_WARN(CCTK_WARN_ABORT, "couldn't get integer array value from table!");
/* now count = 3, blah[0] = 42, blah[1] = 69, blah[2] = 105, */
/*     and all remaining elements of blah[] are unchanged */
\end{verbatim}
As you can see, a table entry remembers the length of any array
value that has been stored in it.%%%
\footnote{%%%
	 In fact, actually \emph{all} table values are
	 arrays---setting or getting a single value is
	 just the special case where the array length is 1.
	 }%%%
{}

If you only want the first few values of a larger array, just pass
in the appropriate length of your array,
that's OK:
\begin{verbatim}
CCTK_INT blah2[2];
int count = Util_TableGetIntArray(handle, 2, blah2, "my array");
if (count < 0)
        CCTK_WARN(CCTK_WARN_ABORT, "couldn't get integer array value from table!");
/* now count = 3, blah2[0] = 42, blah2[1] = 69 */
\end{verbatim}
You can even ask for just the first value:
\begin{verbatim}
CCTK_INT blah1;
int count = Util_TableGetInt(handle, &blah1, "my array");
if (count < 0)
        CCTK_WARN(CCTK_WARN_ABORT, "couldn't get integer array value from table!");
/* now count = 3, blah1 = 42 */
\end{verbatim}

%%%%%%%%%%%%%%%%%%%%%%%%%%%%%%%%%%%%%%%%

\subsection{Character Strings}

One very common thing you might want to store in a table is a
character string.  While you could do this by explicitly storing
an array of \verb|CCTK_CHAR|, there are also routines
specially for conveniently setting and getting strings:
\begin{verbatim}
if (Util_TableSetString(handle, "black holes are fun", "bh") < 0)
        CCTK_WARN(CCTK_WARN_ABORT, "couldn't set string value in table!");

...
char buffer[50];
if (Util_TableGetString(handle, 50, buffer, "bh") < 0)
        CCTK_WARN(CCTK_WARN_ABORT, "couldn't get string value from table!");

/* now buffer[] contains the string "black holes are fun" */
\end{verbatim}

\verb|Util_TableGetString()| guarantees that the string is
terminated by a null character (`\verb|\0|'), and also returns an
error if the string is too long for the buffer.

%%%%%%%%%%%%%%%%%%%%%%%%%%%%%%%%%%%%%%%%

\subsection{Convenience Routines}

There are also convenience routines for the common case of setting
values in a table based on a string.

For example, the following code sets up exactly the same table as the
example in Section \ref{Tables_Simple_Example}:

\begin{verbatim}
#include <util_Table.h>

/* create a table and set some values in it */
int handle = Util_TableCreate(UTIL_TABLE_FLAGS_DEFAULT);
if (handle < 0)
        CCTK_WARN(CCTK_WARN_ABORT, "couldn't create table!");

/* try to set some table entries */
if (Util_TableSetFromString(handle, "two=2 pi=3.14") != 2)
        CCTK_WARN(CCTK_WARN_ABORT, "couldn't set values in table!");
\end{verbatim}

There is also an even higher-level convenience function
\verb|Util_TableCreateFromString()|: this creates a table with the
case insensitive flag set (to match Cactus parameter file semantics),
then (assuming no errors occurred) calls \verb|Util_TableSetFromString()|
to set values in the table.

For example, the following code sets up a table (with the case insensitive flag
set) with four entries: an integer number ({\tt two}), a real number ({\tt
pi}), a string ({\tt buffer}), and an integer array with three elements ({\tt
array}):

\begin{verbatim}
#include <util_Table.h>

int handle = Util_TableCreateFromString(" two    = 2 "
                                        " pi     = 3.14 "
                                        " buffer = 'Hello World' "
                                        " array  = { 1 2 3 }");
if (handle < 0)
        CCTK_WARN(CCTK_WARN_ABORT, "couldn't create table from string!");
\end{verbatim}

Note that this code passes a single string to
\verb|Util_TableCreateFromString()|%%%
\footnote{C automatically concatenates
adjacent character string constants separated only by whitespace.}, 
which then gets parsed into key/value pairs, with the key separated from its
corresponding value by an equals sign.

Values for numbers are converted into integers ({\tt CCTK\_INT}) if possible
(no decimal point appears in the value), otherwise into reals ({\tt CCTK\_REAL}).
Strings must be enclosed in either single or double quotes. String values in
single quotes are interpreted literally, strings in double quotes may contain
character escape codes which then will be interpreted as in C.
Arrays must be enclosed in curly braces, array elements must be single numbers
of the same type (either all integer or all real).

%%%%%%%%%%%%%%%%%%%%%%%%%%%%%%%%%%%%%%%%

\subsection{Table Iterators}
\label{sect-UtilityRoutines/tables/table-iterators}

In the examples up to now, the code, which wanted to get values from
the table, knew what the keys were.  It's also useful to be able to
write generic code which can operate on a table without knowing the
keys.  ``Table iterators'' (``iterators'', for short) are used for this.

An iterator is an abstraction of a pointer to a particular table entry.
Iterators are analogous to the \verb|DIR *| pointers used by the POSIX
\verb|opendir()|, \verb|readdir()|, \verb|closedir()|, and similar
functions, to Perl hash tables' \verb|each()|, \verb|keys()|,
and \verb|values()|, and to the C++ Standard Template Library's
forward iterators.

At any time, the entries in a table may be considered to be in some
arbitrary (implementation-defined) order; an iterator may be used to
walk through some or all of the table entries in this order.  This
order is guaranteed to remain unchanged for any given table, so long
as no changes are made to that table, \ie{} so long as no
\verb|Util_TableSet*()|, \verb|Util_TableSet*Array()|,
\verb|Util_TableSetGeneric()|, \verb|Util_TableSetGenericArray()|,
\verb|Util_TableSetString()|, or \verb|Util_TableDeleteKey()| calls
are made on that table (making such calls on other tables doesn't
matter).  The order may change if there is any change in the table,
and it may differ even between different tables with identical key/value
contents (including those produced by \verb|Util_TableClone()|).%%%
\footnote{%%%
	 For example, if tables were implemented by hashing,
	 the internal order could be that of the hash buckets,
	 and the hash function could depend on the internal
	 table address.
	 }%%%
{}

Any change in the table also invalidates all iterators pointing
anywhere in the table; using any such iterator is an error.
Multiple iterators may point into the same table; they all use the
same order, and (unlike in Perl) they're all independent.

The detailed function description
in the Reference Manual
for \verb|Util_TableItQueryKeyValueInfo()| has an example of
using an iterator to print out all the entries in a table.

%%%%%%%%%%%%%%%%%%%%%%%%%%%%%%%%%%%%%%%%

\subsection{Multithreading and Multiprocessor Issues}

At the moment, the table functions are \emph{not} thread-safe
in a multithreaded environment. 
%% However, this should change in
%%the not-too-distant future: then all the table functions will default
%to being thread-safe.  That is, user code will be able call these
%%functions concurrently from multiple threads, with the table functions
%%automatically doing any necessary locking.%%%
%%\footnote{%%%
%	 For the implementation, this means that we will need a
%	 reader/writer lock for each table and for each iterator:
%	 any number of threads will be able to concurrently read
%	 the data structure, but any write will require exclusive
%	 access.
%	 }%%%
%{}  (We may add a flags-word bit to suppress this for maximum
%performance if you know you won't be making concurrent calls from
%multiple threads.)

Note that tables and iterators are process-wide, i.e. all
threads see the same tables and iterators (think of them as like the
Unix current working directory, with the various routines which modify
the table or change iterators acting like a Unix \verb|chdir()| system
call).

In a multiprocessor environment, tables are always processor-local.

%%%%%%%%%%%%%%%%%%%%%%%%%%%%%%%%%%%%%%%%

\subsection{Metadata about All Tables}

Tables do not \emph{themselves} have names or other
attributes.  However, we may add some special
``system tables'' to be used by Cactus itself to store this sort of
information for those cases where it's needed. For example, we may add support for a
``checkpoint me'' bit in a table's flags word, so that if you want a
table to be checkpointed, you just need to set this bit.
In this case, the table will probably get a system generated name in
the checkpoint dump file.  But if you want the table to have some
other name in the dump file, then you need to tell the checkpointing
code that, by setting an appropriate entry in a checkpoint table.
(You would find the checkpoint table by looking in a special
``master system table'' that records handles of other interesting tables.)

%%%%%%%%%%%%%%%%%%%%%%%%%%%%%%%%%%%%%%%%%%%%%%%%%%%%%%%%%%%%%%%%%%%%%%%%%%%%%%%%
%%%%%%%%%%%%%%%%%%%%%%%%%%%%%%%%%%%%%%%%%%%%%%%%%%%%%%%%%%%%%%%%%%%%%%%%%%%%%%%%
%%%%%%%%%%%%%%%%%%%%%%%%%%%%%%%%%%%%%%%%%%%%%%%%%%%%%%%%%%%%%%%%%%%%%%%%%%%%%%%%


\chapter{Schedule Bins}
\label{sec:Appendix.schedule_bins}

Using the {\tt schedule.ccl} files, thorn functions can be scheduled to run 
in the different timebins which are executed by the Cactus flesh. This chapter
describes these standard timebins, and shows the flow of program execution
through them.

Scheduled functions must be declared as
\begin{Lentry}

\item[In C:]
\begin{verbatim}
#include "cctk_Arguments.h"
void MyFunction (CCTK_ARGUMENTS);
\end{verbatim}

\item[In C++:]
\begin{verbatim}
#include "cctk_Arguments.h"
extern "C" void MyFunction (CCTK_ARGUMENTS);
\end{verbatim}

\item[In Fortran:]
\begin{verbatim}
#include "cctk_Arguments.h"
subroutine MyFunction (CCTK_ARGUMENTS)
   DECLARE_CCTK_ARGUMENTS
end
\end{verbatim}
\end{Lentry}

Exceptions are the functions that are scheduled in the bins {\tt
CCTK\_STARTUP}, {\tt CCTK\_RECOVER\_PARAMETERS}, and {\tt
CCTK\_SHUTDOWN}.  They do not take arguments,
and they return an integer.  They must be declared as
\begin{Lentry}

\item[In C:]
\begin{verbatim}
int MyFunction (void);
\end{verbatim}

\item[In C++]
\begin{verbatim}
extern "C" int MyFunction ();
\end{verbatim}

\item[In Fortran:]
\begin{verbatim}
integer function MyFunction ()
end
\end{verbatim}
\end{Lentry}

The return value in {\tt CCTK\_STARTUP} and {\tt CCTK\_SHUTDOWN} is
unused, and might in the future be used to indicate whether an error
occurred.  You should return 0.

The return value in {\tt CCTK\_RECOVER\_PARAMETERS} should be zero,
positive, or negative, indicating that no parameters were recovered,
that parameters were recovered successfully, or that an error
occurred, respectively.  Routines in this bin are executed in alphabetical
order, according to the owning thorn's name, until one returns a positive
value.  All later routines are ignored.  Schedule clauses \texttt{BEFORE}, 
\texttt{AFTER}, \texttt{WHILE}, \texttt{IF}, etc., are ignored.


\begin{Lentry}

\item[{\tt CCTK\_RECOVER\_PARAMETERS}]
        Used by thorns with relevant I/O methods as the point 
        to read parameters when recovering from checkpoint files.
        Grid variables are not available in this timebin.  Scheduling
	in this timebin is special (see above).

\item[{\tt CCTK\_STARTUP}] 
        Run before any grids are constructed, this is 
        the timebin, for example, where grid independent information 
        (e.g.\ output methods, reduction operators) is registered. 
        Note that since no grids are setup at this point, grid 
        variables cannot be used in routines scheduled here.

\item[{\tt CCTK\_WRAGH}]
        This timebin is executed when all parameters are known, but
        before the driver thorn constructs the grid.  It should only
        be used to set up information that is needed by the driver.

\item[{\tt CCTK\_PARAMCHECK}] 
        This timebin is for thorns to check the validity of
        parameter combinations. This bin is also executed before the
        grid hierarchy is made, so that routines scheduled here only
        have access to the global grid size and the parameters.

\item[{\tt CCTK\_PREREGRIDINITIAL}]
        This timebin is used in mesh refinement settings.  It is
        ignored for unigrid runs.  This bin is executed whenever the
        grid hierarchy is about to change during evolution; compare
        {\tt CCTK\_PREREGRID}.  Routines that decide the
        new grid structure should be scheduled in this bin.

\item[{\tt CCTK\_POSTREGRIDINITIAL}]
        This timebin is used in mesh refinement settings.  It is
        ignored for unigrid runs.  This bin is executed whenever the
        grid hierarchy or patch setup has changed during evolution;
        see {\tt CCTK\_POSTREGRID}.  It is, e.g.
        necessary to re-apply the boundary conditions or recalculate
        the grid points' coordinates after every changing the grid
        hierarchy.

\item[{\tt CCTK\_BASEGRID}]
        This timebin is executed very early after a driver thorn
        constructs grid; this bin should only be used to set up
        coordinate systems on the newly created grids.

\item[{\tt CCTK\_INITIAL}] 
        This is the place to set up any required initial data. This timebin
        is not run when recovering from a checkpoint file.

\item[{\tt CCTK\_POSTINITIAL}]
        This is the place to modify initial data, or to calculate data
        that depend on the initial data.  This timebin is also not run
        when recovering from a checkpoint file.
        
\item[{\tt CCTK\_POSTRESTRICTINITIAL}]
        This timebin is used only in mesh refinement settings.  It is
        ignored for unigrid runs.  This bin is executed after each
        restriction operation while initial data are set up; compare
        {\tt CCTK\_POSTRESTRICT}.  It is,
        e.g. necessary to re-apply the
        boundary conditions after every restriction operation.

\item[{\tt CCTK\_POSTPOSTINITIAL}]
        This is the place to modify initial data, or to calculate data
        that depend on the initial data.  This timebin is executed
        after the recursive initialisation of finer grids if there is
        a mesh refinement hierarchy, and it is also not run
        when recovering from a checkpoint file.

\item[{\tt CCTK\_RECOVER\_VARIABLES}]
        Used by thorns with relevant I/O methods as the point 
        to read in all the grid variables when recovering from 
        checkpoint files.
        
\item[{\tt CCTK\_POST\_RECOVER\_VARIABLES}]
        This timebin exists for scheduling any functions which need 
        to modify grid variables after recovery.

\item[{\tt CCTK\_CPINITIAL}]
        Used by thorns with relevant I/O methods as the point to checkpoint
        initial data if required.

\item[{\tt CCTK\_CHECKPOINT}]
        Used by thorns with relevant I/O methods as the point to checkpoint
        data during the iterative loop when required.

\item[{\tt CCTK\_PREREGRID}]
        This timebin is used in mesh refinement settings.  It is
        ignored for unigrid runs.  This bin is executed whenever the
        grid hierarchy is about to change during evolution; compare
        {\tt CCTK\_PREREGRIDINITIAL}.  Routines that decide the
        new grid structure should be scheduled in this bin.

\item[{\tt CCTK\_POSTREGRID}]
        This timebin is used in mesh refinement settings.  It is
        ignored for unigrid runs.  This bin is executed whenever the
        grid hierarchy or patch setup has changed during evolution;
        see {\tt CCTK\_POSTREGRIDINITIAL}.  It is, e.g.
        necessary to re-apply the boundary conditions or recalculate
        the grid points' coordinates after every changing the grid
        hierarchy.
                
\item[{\tt CCTK\_PRESTEP}]
        The timebin for scheduling any routines which need to be 
        executed before any routines in the main evolution step. This 
        timebin exists for thorn writers convenience, the {\tt BEFORE}, 
        {\tt AFTER}, etc., functionality of the {\tt schedule.ccl} file
        should allow all functions to be scheduled in the main {\tt CCTK\_EVOL}
        timebin.
        
\item[{\tt CCTK\_EVOL}]
        The timebin for the main evolution step.
        
\item[{\tt CCTK\_POSTRESTRICT}]
        This timebin is used only in mesh refinement settings.  It is
        ignored for unigrid runs.  This bin is executed after each
        restriction operation during evolution; compare {\tt
          CCTK\_POSTRESTRICTINITIAL}.  It is, e.g. necessary to
        re-apply the
        boundary conditions after every restriction operation.

\item[{\tt CCTK\_POSTSTEP}]
        The timebin for scheduling any routines which need to be 
        executed after all the routines in the main evolution step. This 
        timebin exists for thorn writers convenience, the {\tt BEFORE}, 
        {\tt AFTER}, etc.,\ functionality of the {\tt schedule.ccl} file
        should allow all functions to be scheduled in the main {\tt CCTK\_EVOL}
        timebin.

\item[{\tt CCTK\_ANALYSIS}]
        The {\tt ANALYSIS} timebin is special, in that it is closely coupled 
        with output, and routines which are scheduled here are typically
        only executed if output of analysis variables is required. 
        Routines which perform analysis should be independent of the main 
        evolution loop (that is, it should not matter for the results
        of a simulation whether routines in this timebin are executed or 
        not). 

\item[{\tt CCTK\_TERMINATE}]
        Called after the main iteration loop when Cactus terminates. 
        Note that sometime, in this timebin, a driver thorn should be 
        destroying the grid hierarchy and removing grid variables. 

\item[{\tt CCTK\_SHUTDOWN}]
        Cactus final shutdown routines, after the grid hierarchy has been 
        destroyed. Grid variables are no longer available.

\end{Lentry}

%%%%%%%%%%%%%%%%%%%%%%%%%%%%%%%%%%%%%%%%%%%%%%%%%%%%%%%%%%%%%%%%%%%%%%%%%%%%%%%%
%%%%%%%%%%%%%%%%%%%%%%%%%%%%%%%%%%%%%%%%%%%%%%%%%%%%%%%%%%%%%%%%%%%%%%%%%%%%%%%%
%%%%%%%%%%%%%%%%%%%%%%%%%%%%%%%%%%%%%%%%%%%%%%%%%%%%%%%%%%%%%%%%%%%%%%%%%%%%%%%%

\chapter{Flesh Parameters}

The flesh parameters are defined in the file {\tt src/param.ccl}.

\section{Private Parameters}

Here, the default value is shown in square brackets, while curly braces show allowed parameter values.

\begin{Lentry}

\item[{\tt allow\_mixeddim\_gfs}]
Allow use of GFs from different dimensions [{\tt no}]

\item [{\tt cctk\_brief\_output}]
Give only brief output [{\tt no}]

\item[{\tt cctk\_full\_warnings}]
Give detailed information for each warning statement [{\tt yes}]

\item [{\tt cctk\_run\_title}]
Description of this simulation [{\tt ""}]

\item [{\tt cctk\_show\_banners}]
Show any registered banners for the different thorns [{\tt yes}]

\item [{\tt cctk\_show\_schedule}]
Print the scheduling tree to standard output [{\tt yes}]

\item[{\tt cctk\_strong\_param\_check}]
Die on parameter errors in {\tt CCTK\_PARAMCHECK} [{\tt yes}]

\item[{\tt cctk\_timer\_output}]
Give timing information [{\tt off}] \{{\tt off, full}\}

\item[{\tt recovery\_mode}]
How to behave when recovering from a checkpoint [{\tt strict}] \{{\tt strict, relaxed}\}

\item[{\tt highlight\_warning\_messages}]
Highlight CCTK warning messages [{\tt yes}]

\item[{\tt info\_format}]
Specifies the content and format of \code{CCTK\_INFO()}/\code{CCTK\_VInfo()}
messages. [{\tt basic}]
% don't want lines broken in the middle of one of the keywords!!
\{%%%
\hbox{{\tt "basic"}},
\hbox{{\tt "numeric time stamp"}},
\hbox{{\tt "human-readable time stamp"}},\\
\hbox{{\tt "full time stamp"}}%%%
\}

\end{Lentry}

\section{Restricted Parameters}

\begin{Lentry}

\item [{\tt cctk\_final\_time}] Final time for evolution, overridden by
{\tt cctk\_itlast} unless it is positive [{\tt -1.0}]

\item[{\tt cctk\_initial\_time}]
Initial time for evolution [{\tt 0.0}]

\item [{\tt cctk\_itlast}]
Final iteration number [{\tt 10}]

\item [{\tt max\_runtime}]
Terminate evolution loop after a certain elapsed runtime (in minutes); set to zero to disable this termination condition [{\tt 0}]

\item [{\tt terminate}]
Condition on which to terminate evolution loop [{\tt iteration}] \{{\tt never, iteration, time, runtime, any, all}\}

\item [{\tt terminate\_next}]
Terminate on next iteration ? [{\tt no}]

\end{Lentry}

%%%%%%%%%%%%%%%%%%%%%%%%%%%%%%%%%%%%%%%%%%%%%%%%%%%%%%%%%%%%%%%%%%%%%%%%%%%%%%%%
%%%%%%%%%%%%%%%%%%%%%%%%%%%%%%%%%%%%%%%%%%%%%%%%%%%%%%%%%%%%%%%%%%%%%%%%%%%%%%%%
%%%%%%%%%%%%%%%%%%%%%%%%%%%%%%%%%%%%%%%%%%%%%%%%%%%%%%%%%%%%%%%%%%%%%%%%%%%%%%%%

\chapter{Using TRAC}
\label{sec:Appendix.trac}

TRAC is a web-based tool for tracking bug reports and feature
requests.  Cactus bugs and feature requests are handled using the TRAC
system hosted by the Einstein Toolkit consortium at
\url{http://trac.einsteintoolkit.org}.  Click on {\em New Ticket} to
create a new ticket in the system.

Here, we briefly describe the main categories when creating a Cactus
problem report.
\begin{Lentry}
\item[{\bf Summary}] A brief and informative subject line.

\item[{\bf Description}] Describe your problem precisely, if you get a
  core dump, include the stack trace, and if possible give the minimal
  number of thorns, this problems occurs with.  Describe how to
  reproduce the problem if it is not clear.  Note that the description
  field (and the comments) allow a wiki-style syntax.  This means that
  blocks of code or error messages should be surrounded by \{\{\{
  ... \}\}\} in order to avoid the text being interpreted as wiki
  markup.  Click on the WikiFormatting link to learn more about the
  available markup.

\item[{\bf Type}] Choose {\em defect} for cases where there is clearly
something wrong and {\em enhancement} for a feature request.

\item[{\bf Priority}] Pick whichever level is appropriate.  {\em
  Blocker} for issues that stop you using the code, {\em critical} for
  very serious problems, {\em major} for things which should
  definitely be addressed, {\em minor} for things which would be good
  to fix but not essential, and {\em optional} for very low priority
  items.  If in doubt, choose either {\em major} or {\em minor}.

\item[{\bf Milestone}] This is used by the maintainers to indicate an
  intention to fix the problem before a particular release of Cactus.

\item[{\bf Component}] Use {\em Cactus} for problems related to the
  Cactus flesh or one of the thorns in one of the Cactus arrangements
  (those in arrangements with names starting ``Cactus'').

\item[{\bf Version}] The Cactus release you are using. You can find
  this out, for example, from an executable by typing {\tt
    cactus\_<config> -v}. % FIXME: We don't list Cactus versions in
                          % TRAC, only ET versions.

\item[{\bf Keywords}] Here you can enter a space-separated list of
  keywords which might be useful for people searching for specific
  types of tickets.  For example, you could enter the thorn name if
  the problem is with a specific thorn, the keyword {\em testsuite} if
  the ticket is related to a test failure, or the keyword {\em
    documentation} if the problem is related to the documentation.

\item[{\bf CC}] Email addresses of people who should be emailed on any
  change to the ticket, such as a comment being added.

\item[{\bf Email or username}] Your email address, so we can get in
  contact with you.
\end{Lentry}

If you have an account on the computer systems at CCT, you can log in
to the TRAC system in order to be recognised.  Otherwise, your
comments will appear as ``anonymous''.

%%%%%%%%%%%%%%%%%%%%%%%%%%%%%%%%%%%%%%%%%%%%%%%%%%%%%%%%%%%%%%%%%%%%%%%%%%%%%%%%
%%%%%%%%%%%%%%%%%%%%%%%%%%%%%%%%%%%%%%%%%%%%%%%%%%%%%%%%%%%%%%%%%%%%%%%%%%%%%%%%
%%%%%%%%%%%%%%%%%%%%%%%%%%%%%%%%%%%%%%%%%%%%%%%%%%%%%%%%%%%%%%%%%%%%%%%%%%%%%%%%

\chapter{Using SVN}
\label{sec:Appendix.svn}
SVN is a version control system, which allows you to  keep
old versions of files (usually source code), log of
when, and why changes occurred, and who made them,  etc.
SVN does not just operate on one file at a
time or one directory at a time,  but
operates  on  hierarchical collections of directories consisting of
version controlled files.  SVN helps to  manage
releases  and  to control the concurrent editing of source
files among multiple  authors. SVN can be obtained from
\url{http://subversion.apache.org}, but is usually available on workstations, or
can be easily installed using a package manager.

An SVN \textit{repository} located on a \textit{server} contains a
hierarchy of directory and files, and any subdirectory can be checked
out independently.  The Cactus flesh and the Cactus arrangements are
organized as repositories on the server {\tt svn.cactuscode.org}.  You
can browse the contents of this repository using a web browser at the
URL \url{http://svn.cactuscode.org}.

You do not need to know about SVN in order to download or update
Cactus using the GetComponents script, though you must have SVN
installed.  In order to contribute changes to Cactus files or your own
thorns, which may also be stored in SVN, you will need a basic
understanding of SVN.  For more information about 

\section{Essential SVN Commands}

Assuming that you have checked out Cactus using the GetComponents
script, the following commands are the minimum you will need in order
to work with SVN in Cactus.

\begin{Lentry}

\item[{\bf svn update}]
Execute  this  command  from  \emph{within}  your  working
directory  when  you  wish  to  update your
copies of source  files  from  changes  that  other
developers have made to the source in the repository.
Merges are performed automatically when possible, a warning is issued
if manual  resolution  is  required  for conflicting  changes. 

\item[{\bf svn add} {\tt file}]
Use this command to enroll new files in SVN records
of your working directory.  The files will be added
to the  repository  the  next  time  you  run  `{\tt svn
commit}'.

\item[{\bf svn commit} {\tt file}]
Use this command to add your local changes to the source to
the repository and, thereby, making it publically available to
checkouts and updates by other users. You cannot commit a
newly created file unless you have \emph{added} it.

\item[{\bf svn diff} {\tt file}]
Show differences between a file in your working directory
and  a file in the source repository, or between two revisions in
source repository.  (Does not change either repository or working
directory.) For example, to see the difference between versions
{\tt 1.8} and {\tt 1.9} of a file {\tt foobar.c}:

{\tt
\begin{verbatim}
svn diff -r 1.8 1.9 foobar.c
\end{verbatim}
}

\item[{\bf svn remove} {\tt file}]
Remove files from the source repository, pending  an {\tt svn commit} on
the same files.

\item[{\bf svn status} {[}file{]}]
This command returns the current status of your local copy relative to
the repository: e.g.\ it indicates local modifications and possible
updates.

\end{Lentry}

For more information about using SVN, you can read the documentation
provided at \url{http://svnbook.red-bean.com}.

%%%%%%%%%%%%%%%%%%%%%%%%%%%%%%%%%%%%%%%%%%%%%%%%%%%%%%%%%%%%%%%%%%%%%%%%%%%%%%%%
%%%%%%%%%%%%%%%%%%%%%%%%%%%%%%%%%%%%%%%%%%%%%%%%%%%%%%%%%%%%%%%%%%%%%%%%%%%%%%%%
%%%%%%%%%%%%%%%%%%%%%%%%%%%%%%%%%%%%%%%%%%%%%%%%%%%%%%%%%%%%%%%%%%%%%%%%%%%%%%%%

\chapter{Using Tags}
\label{sec:Appendix.tags}
Finding your way around in the Cactus structure can be pretty
difficult to handle. To make life easier there is support for \textit{tags},
which lets you browse the code easily from within Emacs/XEmacs or {\tt vi}.
A tags database can be generated with {\tt gmake}:

\section{Tags with Emacs}

The command {\tt gmake TAGS} will create a database for a routine reference
table to be used within Emacs. This database can be accessed within
Emacs if you add either of the following lines to your {\tt .emacs} file:\\
{\tt (setq tags-file-name "CACTUS\_HOME/TAGS")} XOR \\
{\tt (setq tag-table-alist '(("CACTUS\_HOME" . "CACTUS\_HOME/TAGS")))}\\
where {\tt CACTUS\_HOME} is your Cactus directory.\\

You can now easily navigate your Cactus flesh and Toolkits by searching for
functions or ``tags'':
\begin{enumerate}
\item \textbf{ Alt.} will find a tag
\item \textbf{ Alt,} will find the next matching tag
\item \textbf{ Alt*} will go back to the last matched tag
\end{enumerate}
If you add the following lines to your {\tt .emacs} file, the
files found with tags will opened in \emph{read-only} mode:
\begin{verbatim}
(defun find-tag-readonly (&rest a)
  (interactive)
  (call-interactively `find-tag a)
  (if (eq nil buffer-read-only) (setq buffer-read-only t))  )

(defun find-tag-readonly-next (&rest a)
  (interactive)
  (call-interactively `tags-loop-continue a)
  (if (eq nil buffer-read-only) (setq buffer-read-only t))  )

(global-set-key [(control meta \.)] 'find-tag-readonly)
(global-set-key [(control meta \,)] 'find-tag-readonly-next)
\end{verbatim}
The key strokes to use when you want to browse in read-only mode are:
\begin{enumerate}
\item \textbf{CTRL Alt.} will find a tag and open the file in read-only mode
\item \textbf{CTRL Alt,} will find the next matching tag in read-only mode
\end{enumerate}

\section{Tags with \code{vi}}

The commands available are highly dependent upon the version of \code{vi}, but
the following is a selection of commands which may work.


\begin{enumerate}

\item \textbf{vi -t tag}
Start {\tt vi} and position the cursor at the file and line where `tag' is defined.

\item \textbf{Control-{]}}
Find the tag under the cursor.

\item \textbf{:ta tag}
Find a tag.

\item \textbf{:tnext}
Find the next matching tag.

\item \textbf{:pop}
Return to previous location before jump to tag.

\item \textbf{Control-T}
Return to previous location before jump to tag (not widely implemented).

\end{enumerate}

\vspace{1.1cm}

\emph{Note: Currently some of the \texttt{CCTK\_FILEVERSION()} macros
at the top of every source file don't have a trailing semicolon, which
confuses the \texttt{etags} and \texttt{ctags} programs, so tags does not
find the first subroutine in any file with this problem.}

%%%%%%%%%%%%%%%%%%%%%%%%%%%%%%%%%%%%%%%%%%%%%%%%%%%%%%%%%%%%%%%%%%%%%%%%%%%%%%%%
%%%%%%%%%%%%%%%%%%%%%%%%%%%%%%%%%%%%%%%%%%%%%%%%%%%%%%%%%%%%%%%%%%%%%%%%%%%%%%%%
%%%%%%%%%%%%%%%%%%%%%%%%%%%%%%%%%%%%%%%%%%%%%%%%%%%%%%%%%%%%%%%%%%%%%%%%%%%%%%%%

%%%%%%%%%%%%%%%%%%%%%%%%%%%%%%%%%%%%%%%%%%%%%%%%%%%%%%%%%%%%%%%%%%%%%%%%%%%%%%%%
%%%%%%%%%%%%%%%%%%%%%%%%%%%%%%%%%%%%%%%%%%%%%%%%%%%%%%%%%%%%%%%%%%%%%%%%%%%%%%%%
%%%%%%%%%%%%%%%%%%%%%%%%%%%%%%%%%%%%%%%%%%%%%%%%%%%%%%%%%%%%%%%%%%%%%%%%%%%%%%%%

\end{cactuspart}

%%% Local Variables: 
%%% mode: latex
%%% TeX-master: "UsersGuide"
%%% End: 
