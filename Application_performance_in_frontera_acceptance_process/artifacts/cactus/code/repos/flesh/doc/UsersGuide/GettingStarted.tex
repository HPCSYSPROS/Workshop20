% /*@@
%   @file      RunningCactus.tex
%   @date      27 Jan 1999
%   @author    Tom Goodale, Gabrielle Allen, Gerd Lanferman, Thomas Radke
%   @desc
%              How to run Cactus part of the Cactus User's Guide
%   @enddesc
%   @version   $Header$
% @@*/

\begin{cactuspart}{Introduction}{}{$Revision$}\label{part:Introduction}
\renewcommand{\thepage}{\Alph{part}\arabic{page}}

%%%%%%%%%%%%%%%%%%%%%%%%%%%%%%%%%%%%%%%%%%%%%%%%%%%%%%%%%%%%%%%%%%%%%%%%%%%%%%%%
%%%%%%%%%%%%%%%%%%%%%%%%%%%%%%%%%%%%%%%%%%%%%%%%%%%%%%%%%%%%%%%%%%%%%%%%%%%%%%%%
%%%%%%%%%%%%%%%%%%%%%%%%%%%%%%%%%%%%%%%%%%%%%%%%%%%%%%%%%%%%%%%%%%%%%%%%%%%%%%%%

\chapter{Getting Started}
\label{cha:gs}

%%%%%%%%%%%%%%%%%%%%%%%%%%%%%%%%%%%%%%%%%%%%%%%%%%%%%%%%%%%%%%%%%%%%%%%

\section{Obtaining Cactus}
\label{sec:checkout}

Cactus is distributed, extended, and maintained using the free Subversion
software (\url{http://subversion.apache.org/docs/})
SVN allows many people to work on a large software project
together without getting into a tangle.
Since Cactus thorns are distributed from several repositories on the
main SVN site, and from a growing number of user sites, we provide a
\texttt{GetComponents} script on our website for checking out the flesh 
and thorns. The script is available at

\url{https://github.com/gridaphobe/CRL/raw/ET_2011_10/GetComponents}.

The
script takes as an argument the name of a file containing a \textit{ThornList},
that is a list of thorns with the syntax
\begin{alltt}
<\var{arrangement name}>/<\var{thorn name}>
\end{alltt}

Optional directives in the ThornList indicate which repository to
fetch thorns from. The ThornList is written in the {\em Component
  Retrieval Language}, documented at
\url{https://github.com/gridaphobe/CRL/wiki/Component-Retrieval-Language}.

The same script can be used to checkout additional thorns, or to update 
existing ones.

The components that make up Cactus can also be checked out directly
using SVN from \url{http://svn.cactuscode.org}.

Another script, \texttt{MakeThornList}, can be used to produce a minimal
ThornList from a given Cactus par file.  It needs a \emph{master} ThornList
to be copied into your \texttt{Cactus} directory.

See \url{http://cactuscode.org/download/thorns/MakeThornList}.


%%%%%%%%%%%%%%%%%%%%%%%%%%%%%%%%%%%%%%%%%%%%%%%%%%%%%%%%%%%%%%%%%%%%%%%

\subsection{Directory Structure}
\label{sec:dist}

A fresh checkout creates a directory \texttt{Cactus} with the
following subdirectories:

\begin{Lentry}

\item[\texttt{.svn}] the SVN bookkeeping directory, present in every subdirectory

\item[\texttt{doc}] Cactus documentation

\item[\texttt{lib}] contains libraries

\item[\texttt{src}] contains the source code for Cactus

\item [\texttt{arrangements}] contains the Cactus arrangements. The arrangements
  (the actual ``physics'') are not supplied by just checking out just Cactus.
  If the arrangements you want to use are standard Cactus arrangements, or
  reside on our SVN repository (\texttt{svn.cactuscode.org}),
  they can be checked out in similar way to the flesh.
\end{Lentry}

When Cactus is first compiled, it creates a new directory
\texttt{Cactus/configs}, which will contain all the source code, object files
and libraries created during the build process.

Configurations are described in detail in Section \ref{sec:configurations}.

%%%%%%%%%%%%%%%%%%%%%%%%%%%%%%%%%%%%%%%%%%%%%%%%%%%%%%%%%%%%%%%%%%%%%%%

\section{Compiling a Cactus application}

%%%%%%%%%%%%%%%%%%%%%%%%%%%%%%%%%%%%%%%%%%%%%%%%%%%%%%%%%%%%%%%%%%%%%%%


Cactus can be built in different configurations from the same copy of
the source files, and these different configurations coexist in the
\texttt{Cactus/configs} directory. Here are several instances in which
 this can be useful:

\begin{enumerate}
\item{}Different configurations can be for \emph{different
architectures}. You can keep executables for multiple architectures
based on a single copy of source code, shared on a common file
system.
\item{} You can compare different \textit{compiler options}, and \textit{debug-modes}.
  You might want to compile different communication protocols
  (e.g. MPI or Globus), or leave them out all together.
\item{} You can have different configurations for \textit{different thorn
    collections} compiled into your executable.
\end{enumerate}

%%%%%%%%%%%%%%%%%%%%%%%%%%%%%%%%%%%%%%%%%%%%%%%%%%%%%%%%%%%%%%%%%%%%%%%
\subsection{Creating a Configuration}
\label{sec:configurations}

At its simplest, this is done by \texttt{gmake <\var{config}>}.
This generates a
configuration with the name \texttt{\var{config}}, doing its best to
automatically determine the default compilers and compilation flags
suitable for the current architecture.

There are a number of additional command-line arguments which may be supplied
to override some parts of the procedure; they are listed in Section~\ref{sec:confopts}.

Once you have created a new configuration, the command
\\ \\
\texttt{gmake <\var{configuration name}>}
\\ \\
will build an executable, prompting you along the way for the
thorns which should be included. There is a range of \texttt{gmake}
targets and options which are detailed in Section~\ref{sec:gmtafobuanadco}.

%%%%%%%%%%%%%%%%%%%%%%%%%%%%%%%%%%%%%%%%%%%%%%%%%%%%%%%%%%%%%%%%%%%%%%%
%%%%%%%%%%%%%%%%%%%%%%%%%%%%%%%%%%%%%%%%%%%%%%%%%%%%%%%%%%%%%%%%%%%%%%%

\section{Running a Cactus application}

Cactus executables always run from a parameter file (which may be
provided as a command-line argument taken from standard input), which
specifies which thorns to use and sets the values of each thorn's 
parameters (the parameters that are not set will take on default values, 
see~\ref{sec:Appendix.param}).

There is no restriction on the name
of the parameter file, although it is conventional to use the file
extension \texttt{.par}.  Optional command-line arguments can be used
to customise runtime behaviour, and to provide information about the
thorns used in the executable. The general syntax for running Cactus from
a parameter file is then

\texttt{./cactus\_<\var{config}> <\var{parameter file}>
[\var{command-line options}]}

A parameter file is a text file whose lines are either comments
or parameter statements.
Comments are blank lines or lines that begin with either
`\texttt{\#}' or `\texttt{!}'.
A parameter statement consists of one or more parameter names, followed by
an `\texttt{=}', followed by the value(s) for this (these) parameter(s).
Note that all string parameters are case insensitive.

The first parameter statement in any parameter file should set \texttt{ActiveThorns},
which is a special parameter that tells the
program which \textit{thorns} are to be activated.  Only parameters from active
thorns can be set (and only those routines \textit{scheduled} by active thorns
are run).  By default all thorns are inactive. For example, the first
entry in a parameter file which is using just the two thorns
\texttt{CactusPUGH/PUGH} and \texttt{CactusBase/CartGrid3D} should be

\texttt{ActiveThorns = "PUGH CartGrid3D"}

Parameter specifications following \texttt{ActiveThorns} usually are
carried out by listing
the name of the \textit{thorn} which defined the parameter, two colons,
and the name of the parameter --- e.g.~\texttt{wavetoyF77::amplitude}
(see Section~\ref{sec:Cactus_parameters.scope} for more information).

Notes:

\begin{itemize}

\item{} You can obtain lists of the parameters associated with
each thorn using the command-line options \texttt{-o} and \texttt{-O}
(Section~\ref{sec:command_line_options}).

\item{} For examples of parameter files, look in the \texttt{par} directory
        which can be found in most thorns.

\item {}
The Cactus make system provides a mechanism for generating a
\textit{Thorn Guide} containing separate chapters for each thorn and
arrangement in your configuration. 
Details about parameters, grid variables and scheduling are
automatically included in from a thorns CCL files into the Thorn
Guide. To construct a Thorn Guide for the configuration
\texttt{$<$\var{config}$>$} use

\texttt{gmake $<$\var{config}$>$-ThornGuide}

or to make a Thorn Guide for all the thorns in the \texttt{arrangements} directory

\texttt{gmake $<$\var{config}$>$}.

\end{itemize}

%%%%%%%%%%%%%%%%%%%%%%%%%%%%%%%%%%%%%%%%%%%%%%%%%%%%%%%%%%%%%%%%%%%%%%%
%%%%%%%%%%%%%%%%%%%%%%%%%%%%%%%%%%%%%%%%%%%%%%%%%%%%%%%%%%%%%%%%%%%%%%%


\chapter{Getting and looking at output}


\section{Screen output}

As your Cactus executable runs, standard output and standard error
are usually written to the screen. Standard output provides you
with information about the run, and standard error reports warnings
and errors from the flesh and thorns.

As the program runs, the normal output provides the following information:

\begin{Lentry}

\item [Active thorns]
        A report is made as each of the thorns in the \texttt{ActiveThorns}
parameters from the parameter file (see Section~\ref{sec:Parameter_File})
is attempted to be activated. This report
shows whether the thorn activation was successful, and if successful gives the
thorn's implementation. For example

\begin{verbatim}
Activating thorn idscalarwave...Success -> active implementation idscalarwave
\end{verbatim}

\item [Failed parameters]
         If any of the parameters in the parameter file does not belong to any
of the active thorns, or if the parameter value is not in the allowed range
(see Section~\ref{sec:Parameters.Types_and_Ranges}),
an error is registered. For example, if the parameter is not recognised

\begin{verbatim}
Unknown parameter time::ddtfac
\end{verbatim}
or if the parameter value is not in the allowed range

\begin{verbatim}
Unable to set keyword CartGrid3D::type - ByMouth not in any active range
\end{verbatim}

\item [Scheduling information]
        The scheduled routines (see Section~\ref{chap:scheduling}),
are listed, in the order that they will be executed. For example

\begin{verbatim}
----------------------------------------------------------------------
  Startup routines
    Cactus: Register banner for Cactus
    CartGrid3D: Register GH Extension for GridSymmetry
    CartGrid3D: Register coordinates for the Cartesian grid
    IOASCII: Startup routine
    IOBasic: Startup routine
    IOUtil: IOUtil startup routine
    PUGH: Startup routine
    WaveToyC: Register banner

  Parameter checking routines
    CartGrid3D: Check coordinates for CartGrid3D
    IDScalarWave: Check parameters

  Initialisation
    CartGrid3D: Set up spatial 3D Cartesian coordinates on the GH
    PUGH: Report on PUGH set up
    Time: Set timestep based on speed one Courant condition
    WaveToyC: Schedule symmetries
    IDScalarWave: Initial data for 3D wave equation

  do loop over timesteps
    WaveToyC: Evolution of 3D wave equation
    t = t+dt
    if (analysis)
    endif
  enddo
----------------------------------------------------------------------
\end{verbatim}

\item [Thorn banners]
        Usually a thorn registers a short piece of text as a \emph{banner}.
        This banner of each thorn is displayed in the standard output when
        the thorn is initialised.

\end{Lentry}


\section{File output}
Output methods in Cactus are all provided by thorns.
Any number of output methods can be used for each run.
The behaviour of the output thorns in the
standard arrangements are described in those thorns' documentation.

In general, output thorns decide what to output by parsing a string parameter
containing the names of those grid variables, or groups of variables, for which
output is required. The names should be fully qualified with the
implementation and group or variable names.

There is usually a parameter for each method to denote how often, in evolution
iterations, this output should be performed.  There is also usually a parameter
to define the directory in which the output should be placed, defaulting to the
directory from which the executable is run.

See Chapter~\ref{chap:io_methods} for details on creating your own IO method.


%%%%%%%%%%%%%%%%%%%%%%%%%%%%%%%%%%%%%%%%%%%%%%%%%%%%%%%%%%%%%%%%%%%%%%%

\chapter{Checkpointing/Recovery}
\label{sec:checkpointing}

Checkpointing is defined as saving the current state of a run (parameter
settings, contents of grid variables, and other relevant information) to a file.
At a later time, this run can then be restarted from that state by recovering
all the data from the checkpoint file.

Cactus checkpointing and recovery methods are provided by thorns.
In general, these thorns decide how often to generate a checkpoint.
They also register their recovery routines with the flesh; these recovery
routines may then be called during initialisation of a subsequent run to
perform the recovery of the state of the run.
Such a recovery is requested by setting a parameter in the parameter file.

See Chapter~\ref{chap:cp_recovery_methods} for details of how to create
your own checkpointing and recovery methods.

%%%%%%%%%%%%%%%%%%%%%%%%%%%%%%%%%%%%%%%%%%%%%%%%%%%%%%%%%%%%%%%%%%%%%%%%%%%%%%%%
%%%%%%%%%%%%%%%%%%%%%%%%%%%%%%%%%%%%%%%%%%%%%%%%%%%%%%%%%%%%%%%%%%%%%%%%%%%%%%%%
%%%%%%%%%%%%%%%%%%%%%%%%%%%%%%%%%%%%%%%%%%%%%%%%%%%%%%%%%%%%%%%%%%%%%%%%%%%%%%%%

%%%%%%%%%%%%%%%%%%%%%%%%%%%%%%%%%%%%%%%%%%%%%%%%%%%%%%%%%%%%%%%%%%%%%%%

\chapter{Reporting bugs}
\label{sec:gehe}

For tracking problem reports and bugs, we use the TRAC bug tracking
system located at \url{http://trac.einsteintoolkit.org} which allows
easy submission and browsing of problem tickets.

A description of the TRAC categories we use is provided in the Appendix
\ref{sec:Appendix.trac}.

%%%%%%%%%%%%%%%%%%%%%%%%%%%%%%%%%%%%%%%%%%%%%%%%%%%%%%%%%%%%%%%%%%%%%%%

%%%% FIXME: add something quick on required software and supported architectures

%%%% FIXME: add an example

\end{cactuspart}
