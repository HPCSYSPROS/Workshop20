% *======================================================================*
%  Cactus Thorn template for ThornGuide documentation
%  Author: Ian Kelley
%  Date: Sun Jun 02, 2002
%  $Header$
%
%  Thorn documentation in the latex file doc/documentation.tex
%  will be included in ThornGuides built with the Cactus make system.
%  The scripts employed by the make system automatically include
%  pages about variables, parameters and scheduling parsed from the
%  relevent thorn CCL files.
%
%  This template contains guidelines which help to assure that your
%  documentation will be correctly added to ThornGuides. More
%  information is available in the Cactus UsersGuide.
%
%  Guidelines:
%   - Do not change anything before the line
%       % BEGIN CACTUS THORNGUIDE",
%     except for filling in the title, author, date etc. fields.
%   - You can define your own macros are OK, but they must appear after
%     the BEGIN CACTUS THORNGUIDE line, and do not redefine standard
%     latex commands.
%   - To avoid name clashes with other thorns, 'labels', 'citations',
%     'references', and 'image' names should conform to the following
%     convention:
%       ARRANGEMENT_THORN_LABEL
%     For example, an image wave.eps in the arrangement CactusWave and
%     thorn WaveToyC should be renamed to CactusWave_WaveToyC_wave.eps
%   - Graphics should only be included using the graphix package.
%     More specifically, with the "includegraphics" command. Do
%     not specify any graphic file extensions in your .tex file. This
%     will allow us (later) to create a PDF version of the ThornGuide
%     via pdflatex. |
%   - References should be included with the latex "bibitem" command.
%   - For the benefit of our Perl scripts, and for future extensions,
%     please use simple latex.
%
% *======================================================================*
%
% Example of including a graphic image:
%    \begin{figure}[ht]
% 	\begin{center}
%    	   \includegraphics[width=6cm]{MyArrangement_MyThorn_MyFigure}
% 	\end{center}
% 	\caption{Illustration of this and that}
% 	\label{MyArrangement_MyThorn_MyLabel}
%    \end{figure}
%
% Example of using a label:
%   \label{MyArrangement_MyThorn_MyLabel}
%
% Example of a citation:
%    \cite{MyArrangement_MyThorn_Author99}
%
% Example of including a reference
%   \bibitem{MyArrangement_MyThorn_Author99}
%   {J. Author, {\em The Title of the Book, Journal, or periodical}, 1 (1999),
%   1--16. {\tt http://www.nowhere.com/}}
%
% *======================================================================*

% If you are using CVS use this line to give version information
% $Header$

\documentclass{article}

% Use the Cactus ThornGuide style file
% (Automatically used from Cactus distribution, if you have a
%  thorn without the Cactus Flesh download this from the Cactus
%  homepage at www.cactuscode.org)
\usepackage{../../../../doc/latex/cactus}

\begin{document}

%%%%%%%%%%%%%%%%%%%%%%%%%%%%%%%%%%%%%%%%%%%%%%%%%%%%%%%%%%%%%%%%%%%%%%%%%%%%%%%%

% The author of the documentation
\author{Thomas Radke}

% The title of the document (not necessarily the name of the Thorn)
\title{Thorn Guide for the {\bf LocalInterp} Thorn}

% the date your document was last changed, if your document is in CVS,
% please us:
%    \date{$ $Date$ $}
\date{$ $Date$ $}

\maketitle

% Do not delete next line
% START CACTUS THORNGUIDE

% Add all definitions used in this documentation here
%   \def\mydef etc

% Add an abstract for this thorn's documentation
\begin{abstract}
This thorn does processor-local interpolation of N-dimensional data
arrays.  In general there may be many input arrays (all defined on the
same uniform Cartesian grid) all being interpolated to the same set
of interpolation points.
\end{abstract}

% The following sections are suggestive only.
% Remove them or add your own.

%%%%%%%%%%%%%%%%%%%%%%%%%%%%%%%%%%%%%%%%%%%%%%%%%%%%%%%%%%%%%%%%%%%%%%%%%%%%%%%%

\section{Introduction}

This thorn provides processor-local interpolation, using the interpolation
operator
\begin{verbatim}
  "uniform cartesian"
\end{verbatim}
for the Cactus local interpolation API \verb|CCTK_InterpLocalUniform()|.
(Note that the word ``{\tt cartesian}'' is in lower case here!)
It supports 1, 2, and 3-dimensional interpolation.

See the Cactus Reference Manual
for a full description of the \verb|CCTK_InterpLocalUniform()| API,
and some examples of how to use it.

%%%%%%%%%%%%%%%%%%%%%%%%%%%%%%%%%%%%%%%%

\subsection{History}

This interpolator was written by Thomas Radke in early 2001 (drawing
on older code by Paul Walker).  It originally lived in the PUGHInterp
thorn, but it turned to have very little to do with PUGH, so was moved
here in winter 2001-2002.

From winter 2001-2002 to July 2003 this thorn also contained another
interpolator written by Jonathan Thornburg, but in July 2003 that
interpolator was moved to {\bf AEIThorns/AEILocalInterp/} because it was
(is) GPL and Cactus policies are that this arrangement ({\bf CactusBase})
is reserved for code under the Cactus-flesh license (= GPL except
that it's like LGPL for linking with other thorns).

At the beginning of 2015 the low-level interpolation kernels of this
thorn were rewritten in C++ by David Radice.
%%%%%%%%%%%%%%%%%%%%%%%%%%%%%%%%%%%%%%%%%%%%%%%%%%%%%%%%%%%%%%%%%%%%%%%%%%%%%%%%

\section{Implementation}

Internally, this interpolator always does 3-D interpolation, inserting
zero coordinates as appropriate for lower dimensionalities.  The
interpolation is done by successive 1-D interpolations along each
axis.%%%
\footnote{%%%
	 Note that this means that different axes are treated
	 slightly differently by the interpolator.  In other
	 words, at the level of finite differencing errors,
	 interpolation does {\em not\/} commute with permuting
	 the axes.  However, in practice the differences are
	 likely to be small, at least for smooth input data.
	 }%%%
{}  See the \verb|README| file in the \verb|src/| directory
for further details.

\subsection{Additional information passed in as table options}

The \verb|CCTK_InterpLocalUniform()| API accepts a table handle as one of its
arguments which can be used to pass additional information to the local
interpolator via table options.

The only table option supported so far by {\bf LocalInterp}'s {\tt "uniform
cartesian"} operator is the interpolation order which must be
passed as a CCTK\_INT value with key {\tt "order"}. Options with keys
{\tt "N\_boundary\_points\_to\_omit", "boundary\_off\_centering\_tolerance"}, or
{\tt "boundary\_extrapolation\_tolerance"} (which are usually passed by a
global interpolator) are also recognized but silently ignored (a level-4 warning
message will be issued in this case).

%%%%%%%%%%%%%%%%%%%%%%%%%%%%%%%%%%%%%%%%%%%%%%%%%%%%%%%%%%%%%%%%%%%%%%%%%%%%%%%%

% Do not delete next line
% END CACTUS THORNGUIDE

\end{document}
