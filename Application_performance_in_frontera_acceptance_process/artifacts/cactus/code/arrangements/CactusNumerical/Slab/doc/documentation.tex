% *======================================================================*
%  Cactus Thorn template for ThornGuide documentation
%  Author: Ian Kelley
%  Date: Sun Jun 02, 2002
%
%  Thorn documentation in the latex file doc/documentation.tex 
%  will be included in ThornGuides built with the Cactus make system.
%  The scripts employed by the make system automatically include 
%  pages about variables, parameters and scheduling parsed from the 
%  relevent thorn CCL files.
%  
%  This template contains guidelines which help to assure that your     
%  documentation will be correctly added to ThornGuides. More 
%  information is available in the Cactus UsersGuide.
%                                                    
%  Guidelines:
%   - Do not change anything before the line
%       % START CACTUS THORNGUIDE",
%     except for filling in the title, author, date etc. fields.
%        - Each of these fields should only be on ONE line.
%        - Author names should be sparated with a \\ or a comma
%   - You can define your own macros are OK, but they must appear after
%     the START CACTUS THORNGUIDE line, and do not redefine standard 
%     latex commands.
%   - To avoid name clashes with other thorns, 'labels', 'citations', 
%     'references', and 'image' names should conform to the following 
%     convention:          
%       ARRANGEMENT_THORN_LABEL
%     For example, an image wave.eps in the arrangement CactusWave and 
%     thorn WaveToyC should be renamed to CactusWave_WaveToyC_wave.eps
%   - Graphics should only be included using the graphix package. 
%     More specifically, with the "includegraphics" command. Do
%     not specify any graphic file extensions in your .tex file. This 
%     will allow us (later) to create a PDF version of the ThornGuide
%     via pdflatex. |
%   - References should be included with the latex "bibitem" command. 
%   - use \begin{abstract}...\end{abstract} instead of \abstract{...}
%   - For the benefit of our Perl scripts, and for future extensions, 
%     please use simple latex.     
%
% *======================================================================* 
% 
% Example of including a graphic image:
%    \begin{figure}[ht]
% 	\begin{center}
%    	   \includegraphics[width=6cm]{MyArrangement_MyThorn_MyFigure}
% 	\end{center}
% 	\caption{Illustration of this and that}
% 	\label{MyArrangement_MyThorn_MyLabel}
%    \end{figure}
%
% Example of using a label:
%   \label{MyArrangement_MyThorn_MyLabel}
%
% Example of a citation:
%    \cite{MyArrangement_MyThorn_Author99}
%
% Example of including a reference
%   \bibitem{MyArrangement_MyThorn_Author99}
%   {J. Author, {\em The Title of the Book, Journal, or periodical}, 1 (1999), 
%   1--16. {\tt http://www.nowhere.com/}}
%
% *======================================================================* 

% If you are using CVS use this line to give version information

\documentclass{article}

% Use the Cactus ThornGuide style file
% (Automatically used from Cactus distribution, if you have a 
%  thorn without the Cactus Flesh download this from the Cactus
%  homepage at www.cactuscode.org)
\usepackage{../../../../doc/latex/cactus}

\begin{document}

% The author of the documentation
\author{Erik Schnetter \textless eschnetter@perimeterinstitute.ca\textgreater}

% The title of the document (not necessarily the name of the Thorn)
\title{Generic Hyperslabbing}

% the date your document was last changed, if your document is in CVS, 
% please use:
\date{$ $Date$ $}

\maketitle

% Do not delete next line
% START CACTUS THORNGUIDE

% Add all definitions used in this documentation here 
%   \def\mydef etc

% Add an abstract for this thorn's documentation
\begin{abstract}
The Slab thorn provides generic slabbing facilities.  A slab is a
sub-array of another array.  Both the array and the slab can be
multidimensional, and the slab can have a non-unit stride.  The Slab
thorn provides a routine to copy a slab from one array into a slab of
another array, while possibly transposing or mirroring the slab.  The
processor distribution of the arrays can be specified freely, so that
the Slab thorn can be used to interface to non-Cactus libraries with
different data layouts.

The Slab thorn is driver independent, i.e., not tied to PUGH or
Carpet, and does not require MPI for single-processor configurations.
\end{abstract}

% The following sections are suggestive only.
% Remove them or add your own.



\section{Introduction}

A \emph{Slab} is a subarray of another array.  This concept is used in
many places with many different names.  Fortran has so-called ``array
subscript triplets'', which represent the same thing.  In BLAS, the
``leading dimension'' arguments are used to described slabs.  Slabs
are sometimes also called ``array views'' in object oriented
applications.  Slabs are rectangular in shape, and can be rotated with
regard to their containing array by multiples of 90 degrees.  They can
also be mirrored, i.e., the direction of axes can be inverted.

It is often necessary to copy slabs from one array into other arrays,
or to copy one slab of an array to another slab of the same array.
This can be used to change the processor distribution of some data, or
to apply symmetry or periodicity boundary conditions, or to collect
data onto a single processor to process it more easily.



\section{Using This Thorn}

\begin{FunctionDescription}{Slab\_Transfer}{}
Transfer a slab contained in one array to a (possibly different) slab
of another (possibly the same) array

\begin{SynopsisSection}
\begin{Synopsis}{C}
\begin{verbatim}
INHERITS: Slab
USES INCLUDE HEADER: Slab.h

#include "cctk.h"
#include "Slab.h"

struct slabinfo {
  int gsh;
  int lbnd, lsh;
  int ash;
  int lbbox, ubbox, nghostzones;
  int off, str, len;
};

struct xferinfo {
  struct slabinfo src, dst;
  int xpose;
  int flip;
};

int Slab_Transfer (cGH                   * restrict const cctkGH,
                   int                              const dim,
                   struct xferinfo const * restrict const xferinfo,
                   int                              const options,
                   int                              const srctype,
                   void            const *          const srcptr,
                   int                              const dsttype,
                   void                  *          const dstptr);
\end{verbatim}
\end{Synopsis}
\end{SynopsisSection}

\begin{ResultSection}
\begin{Result}{0}
Success
\end{Result}
\begin{Result}{nonzero}
Failure
\end{Result}
\end{ResultSection}

\begin{ParameterSection}
\begin{Parameter}{cctkGH}
Pointer to the CCTK grid hierarchy
\end{Parameter}

\begin{Parameter}{dim}
Number of dimensions of the arrays and slabs.  Must be nonnegative.
\end{Parameter}

\begin{Parameter}{xferinfo[dim]}
Describes the layout of the slab transfer, i.e., the shape and
distribution of the source and destination arrays, and the locations
of the source and destination slabs, and a possible transformation
between the slabs.  Each dimension is described separately.  See the
entries \texttt{xferinfo[d].*} below.
\end{Parameter}

\begin{Parameter}{xferinfo[d].src}
Describes the source array and source slab.  See the \textit{slabinfo}
entries below.
\end{Parameter}

\begin{Parameter}{xferinfo[d].dst}
Describes the destination array and destination slab.  See the
\textit{slabinfo} entries below.
\end{Parameter}

\begin{Parameter}{xferinfo[d].xpose}
Describes how to transpose the slab, i.e., possibly permuting the slab
axes, as in $(x,y) \rightarrow (y,x)$.  \texttt{xferinfo[d].xpose}
contains an integer value in the range \texttt{0\ldots dim-1},
specifying the source axis corresponding to the destination axis
\texttt{d}.  Specify \texttt{xferinfo[d].xpose = d} for no
transposition.  No two values of \texttt{xferinfo[*].xpose} may be the
same.
\end{Parameter}

\begin{Parameter}{xferinfo[d].flip}
Describes how to mirror the slab, i.e., possibly inverting the slab
axes, as in $(x) \rightarrow (-x)$.  \texttt{xferinfo[d].flip}
containes a boolean value specifying whether the axis in direction
\texttt{d} should be inverted, i.e., either \texttt{0} or \texttt{1},
where \texttt{0} indicates no inversion, and \texttt{1} indicates
inversion.

When axes are both transposed and inverted while a slab is copied,
then the transposing happens first, and the axis inversion later.
That is, the sequence of ``actions'' is: extract slab, transpose,
invert, insert slab.  For example, when transposing the $x$ and $z$
axes and inverting the $x$ axis, then the destination slab's $x$ axis
is the source slab's flipped $z$ axis, while the destination $z$ axis
is the unflipped source $x$ axis.
\end{Parameter}

\begin{Parameter}{\textit{slab}}
Describes the shape and processor distribution of one dimension of an
array, and the location of this dimension of a slab.  The shape and
distribution is specified in the same manner as in the \texttt{cGH}
structure.  See the entries \texttt{\textit{slab}.*} below.
\end{Parameter}

\begin{Parameter}{\textit{slab}.gsh}
Global shape of the array; the overall number of grid points
\end{Parameter}

\begin{Parameter}{\textit{slab}.lbnd}
Lower boundary of the array; the global index of the lower boundary of
the processor-local part of the array
\end{Parameter}

\begin{Parameter}{\textit{slab}.lsh}
Local shape of the array; the number of grid points on this processor
\end{Parameter}

\begin{Parameter}{\textit{slab}.ash}
Local allocated shape of the array; the number of grid points on this
processor including padding
\end{Parameter}

\begin{Parameter}{\textit{slab}.lbbox}
Lower bounding box of the array; whether the lower boundary of the
array is an outer boundary.  This corresponds to the even entries in
Cactus' \texttt{bbox} array.  Must be \texttt{0} or \texttt{1}.
\end{Parameter}

\begin{Parameter}{\textit{slab}.ubbox}
Upper bounding box of the array; whether the upper boundary of the
array is an outer boundary.  This corresponds to the odd entries in
Cactus' \texttt{bbox} array.  Must be \texttt{0} or \texttt{1}.
\end{Parameter}

\begin{Parameter}{\textit{slab}.nghostzones}
Number of ghost zones of the array; the number of grid points on this
processor that are only copied from neighbouring processors
\end{Parameter}

\begin{Parameter}{\textit{slab}.off}
Slab offset; the global index of the lowest grid point of the slab
\end{Parameter}

\begin{Parameter}{\textit{slab}.str}
Slab stride; the distance between two grid points making up the slab.
Specify \texttt{\textit{slab}.str = 1} for a unit stride.  Must be
positive.
\end{Parameter}

\begin{Parameter}{\textit{slab}.len}
Slab length; the number of grid points that make up the slab.  This
does not count the grid points that are skipped if the slab has a
non-unit stride.  Must be nonnegative.
\end{Parameter}

\begin{Parameter}{options}
Handle of an option table.  Two options are understood,
\texttt{CCTK\_INT useghosts} and \texttt{CCTK\_POINTER comm}.
\texttt{useghosts} (see below) specifies whether values in ghost zones
may be used in the source array.  \texttt{comm} is a pointer to an MPI
communicator that should be used instead of \texttt{MPI\_COMM\_WORLD}
or the communicator returned by the aliased function
\texttt{GetMPICommWorld}.
\end{Parameter}

\begin{Parameter}{srctype}
Type of the source array.  Pass the corresponding
\texttt{CCTK\_VARIABLE\_*} constant.
\end{Parameter}

\begin{Parameter}{srcptr}
Pointer to the source array
\end{Parameter}

\begin{Parameter}{dsttype}
Type of the destination array.  Pass the corresponding
\texttt{CCTK\_VARIABLE\_*} constant.
\end{Parameter}

\begin{Parameter}{dstptr}
Pointer to the destination array
\end{Parameter}
\end{ParameterSection}

\begin{Discussion}
\texttt{Slab\_Transfer} copies one slab from one array onto a possibly
different slab in possibly the same array.  The shape and processor
distribution of the arrays can be specified freely, as long as each
processor holds a rectangular subset of grid points.  The location of
the slab can also be specified freely, and the slab can be rotated (by
multiples of 90 degrees) or inverted before it is copied.  The source
and destination slab are allowed to overlap.

By default, \texttt{Slab\_Transfer} is conservative with regard to
synchronisation.  It assumes that the ghost zones of the source array
are not valid, but will correctly fill in the ghost zones in the
destination slab.  This can be changed by setting the option
\texttt{useghosts} in the option table.

There are currently some restrictions, which can be remedied if the
need arises: The dimension must be 3 (lower dimensions can easily be
padded).  The communication schedule is set up for every slab
transfer, which is expensive.  The number of ghost zones along the
lower and upper boundary is assumed to be the same.  There is no
Fortran interface.
\end{Discussion}

\begin{SeeAlsoSection}
\begin{SeeAlso}{CarpetSlab}
The hyperslabbing thorn of the driver Carpet.
\end{SeeAlso}

\begin{SeeAlso}{PUGHSlab}
The hyperslabbing thorn of the driver CactusPUGH.
\end{SeeAlso}

\begin{SeeAlso}{New hyperslabbing API}
The web page
\texttt{http://www.cactuscode.org/Development/Current.html} contains a
hyperlink to the new proposed hyperslabbing API\@.  This API is slightly
different from the one used here.
\end{SeeAlso}
\end{SeeAlsoSection}

\begin{ExampleSection}
\begin{Example}{C}
The identity transfer: copy a whole 3D grid function without
transforming it

\begin{verbatim}
INHERITS: Slab
USES INCLUDE HEADER: Slab.h

#include <assert.h>
#include "cctk.h"
#include "cctk_Arguments.h"

#include "Slab.h"

DECLARE_CCTK_ARGUMENTS;

struct xferinfo info[3];
int d;
int ierr;

/* Set up the array descriptors */
assert (cctk_dim <= 3);
for (d=0; d<cctk_dim; ++d) {
  info[d].src.gsh = cctk_gsh[d];
  info[d].src.lbnd = cctk_lbnd[d];
  info[d].src.lsh = cctk_lsh[d];
  info[d].src.ash = cctk_ash[d];
  info[d].src.lbbox = cctk_bbox[2*d];
  info[d].src.ubbox = cctk_bbox[2*d+1];
  info[d].src.nghostzones = cctk_nghostzones[d];
  info[d].dst.gsh = cctk_gsh[d];
  info[d].dst.lbnd = cctk_lbnd[d];
  info[d].dst.lsh = cctk_lsh[d];
  info[d].dst.ash = cctk_ash[d];
  info[d].dst.lbbox = cctk_bbox[2*d];
  info[d].dst.ubbox = cctk_bbox[2*d+1];
  info[d].dst.nghostzones = cctk_nghostzones[d];
}

/* Set up the slab and transformation descriptors */
assert (cctk_dim <= 3);
for (d=0; d<cctk_dim; ++d) {
  info[d].src.off = 0;
  info[d].src.len = cctk_gsh[d];
  info[d].src.str = 1;
  info[d].dst.off = 0;
  info[d].dst.len = cctk_gsh[d];
  info[d].dst.str = 1;
  info[d].xpose = d;
  info[d].flip = 0;
}

/* Transfer the slab */
ierr = Slab_Transfer
  (cctkGH, cctk_dim, info, -1,
   CCTK_VARIABLE_REAL, gf1, CCTK_VARIABLE_REAL, gf2);
assert (!ierr);
\end{verbatim}
\end{Example}

\begin{Example}{C}
A complicated transfer: copy a subarray, transposing the $x$ and $z$
axes, and inverting the $x$ axis

\begin{verbatim}
INHERITS: Slab
USES INCLUDE HEADER: Slab.h

#include <assert.h>
#include "cctk.h"
#include "cctk_Arguments.h"

#include "Slab.h"

DECLARE_CCTK_ARGUMENTS;

struct xferinfo info[3];
int d;
int ierr;

/* Set up the array descriptors (same as above) */
assert (cctk_dim <= 3);
for (d=0; d<cctk_dim; ++d) {
  info[d].src.gsh = cctk_gsh[d];
  info[d].src.lbnd = cctk_lbnd[d];
  info[d].src.lsh = cctk_lsh[d];
  info[d].src.ash = cctk_ash[d];
  info[d].src.lbbox = cctk_bbox[2*d];
  info[d].src.ubbox = cctk_bbox[2*d+1];
  info[d].src.nghostzones = cctk_nghostzones[d];
  info[d].dst.gsh = cctk_gsh[d];
  info[d].dst.lbnd = cctk_lbnd[d];
  info[d].dst.lsh = cctk_lsh[d];
  info[d].dst.ash = cctk_ash[d];
  info[d].dst.lbbox = cctk_bbox[2*d];
  info[d].dst.ubbox = cctk_bbox[2*d+1];
  info[d].dst.nghostzones = cctk_nghostzones[d];
}

/* Set up the slab and transformation descriptors */
info[0].src.off = 5;
info[1].src.off = 5;
info[2].src.off = 1;
info[0].src.str = 1;
info[1].src.str = 1;
info[2].src.str = 1;
info[0].src.len = 2;
info[1].src.len = 3;
info[2].src.len = 2;

info[0].dst.off = 2;
info[1].dst.off = 3;
info[2].dst.off = 4;
info[0].dst.str = 1;
info[1].dst.str = 1;
info[2].dst.str = 1;
info[0].dst.len = 2;
info[1].dst.len = 3;
info[2].dst.len = 2;

info[0].xpose = 2;
info[1].xpose = 1;
info[2].xpose = 0;
info[0].flip = 1;
info[1].flip = 0;
info[2].flip = 0;

/* Transfer the slab */
ierr = Slab_Transfer
  (cctkGH, cctk_dim, info, -1,
   CCTK_VARIABLE_REAL, gf1, CCTK_VARIABLE_REAL, gf2);
assert (!ierr);
\end{verbatim}
\end{Example}
\end{ExampleSection}

\end{FunctionDescription}



\section{Obtaining This Thorn}

This thorn is available from the \texttt{CactusNumerical} arrangement in the
\texttt{arrangements} directory on the \texttt{svn.cactuscode.org}
CVS server via anonymous pserver access.


\section{Support and Feedback}

This thorn was written by Erik Schnetter \textless
eschnetter@perimeterinstitute.ca\textgreater\ in 2002.  Contact me for
questions and comments.

% Do not delete next line
% END CACTUS THORNGUIDE

\end{document}
