% *======================================================================*
%  Cactus Thorn template for ThornGuide documentation
%  Author: Ian Kelley
%  Date: Sun Jun 02, 2002
%  $Header$
%
%  Thorn documentation in the latex file doc/documentation.tex
%  will be included in ThornGuides built with the Cactus make system.
%  The scripts employed by the make system automatically include
%  pages about variables, parameters and scheduling parsed from the
%  relevant thorn CCL files.
%
%  This template contains guidelines which help to assure that your
%  documentation will be correctly added to ThornGuides. More
%  information is available in the Cactus UsersGuide.
%
%  Guidelines:
%   - Do not change anything before the line
%       % START CACTUS THORNGUIDE",
%     except for filling in the title, author, date, etc. fields.
%        - Each of these fields should only be on ONE line.
%        - Author names should be separated with a \\ or a comma.
%   - You can define your own macros, but they must appear after
%     the START CACTUS THORNGUIDE line, and must not redefine standard
%     latex commands.
%   - To avoid name clashes with other thorns, 'labels', 'citations',
%     'references', and 'image' names should conform to the following
%     convention:
%       ARRANGEMENT_THORN_LABEL
%     For example, an image wave.eps in the arrangement CactusWave and
%     thorn WaveToyC should be renamed to CactusWave_WaveToyC_wave.eps
%   - Graphics should only be included using the graphicx package.
%     More specifically, with the "\includegraphics" command.  Do
%     not specify any graphic file extensions in your .tex file. This
%     will allow us to create a PDF version of the ThornGuide
%     via pdflatex.
%   - References should be included with the latex "\bibitem" command.
%   - Use \begin{abstract}...\end{abstract} instead of \abstract{...}
%   - Do not use \appendix, instead include any appendices you need as
%     standard sections.
%   - For the benefit of our Perl scripts, and for future extensions,
%     please use simple latex.
%
% *======================================================================*
%
% Example of including a graphic image:
%    \begin{figure}[ht]
% 	\begin{center}
%    	   \includegraphics[width=6cm]{MyArrangement_MyThorn_MyFigure}
% 	\end{center}
% 	\caption{Illustration of this and that}
% 	\label{MyArrangement_MyThorn_MyLabel}
%    \end{figure}
%
% Example of using a label:
%   \label{MyArrangement_MyThorn_MyLabel}
%
% Example of a citation:
%    \cite{MyArrangement_MyThorn_Author99}
%
% Example of including a reference
%   \bibitem{MyArrangement_MyThorn_Author99}
%   {J. Author, {\em The Title of the Book, Journal, or periodical}, 1 (1999),
%   1--16. {\tt http://www.nowhere.com/}}
%
% *======================================================================*

% If you are using CVS use this line to give version information
% $Header$

\documentclass{article}

% Use the Cactus ThornGuide style file
% (Automatically used from Cactus distribution, if you have a
%  thorn without the Cactus Flesh download this from the Cactus
%  homepage at www.cactuscode.org)
\usepackage{../../../../doc/latex/cactus}
\usepackage{color}

\begin{document}

% The author of the documentation
\author{Roland Haas \textless rhaas@tapir.caltech.edu\textgreater}

% The title of the document (not necessarily the name of the Thorn)
\title{Controlling Cactus parameters during core collapse simulations}

% the date your document was last changed, if your document is in CVS,
% please use:
%    \date{$ $Date: 2004-01-07 12:12:39 -0800 (Wed, 07 Jan 2004) $ $}
\date{April 04 2013}

\maketitle

% Do not delete next line
% START CACTUS THORNGUIDE

% Add all definitions used in this documentation here
%   \def\mydef etc
\newcommand{\todo}[1]{{\color{blue}TODO:#1}}

% Add an abstract for this thorn's documentation
\begin{abstract}
Simulating a collapsing star there are often several distinct ``stages'' to
the collapse, each of which requires slightly different settings for the
Cactus parameters. CoreCollaspeControl provides a high level interface to
detect which stage the collaspe is in an to steer a variety of parameters
based on the stage. CoreCollaspeControl provides support to control
CarpetRegrid2 by successively activating refinement levels as the collapse
provgresses, CarpetIOScalar, CarpetIOASCII and CarpetIOHDF5 to control output
frequency, various gravitational wave extraction thorns
(ZelmaniQuadWaveExtract, InnerCore, ZerilliEF, WaveExtract, NPScalars) and
analysis thorns (InnerCore, AHFinderDirect). 
\end{abstract}

% The following sections are suggestive only.
% Remove them or add your own.

\section{Introduction}
At the begiinig of a core collaspe simulation that starts from a collapsing
steller the star collapses smoothly gradually increasing the density and mass
of the inner core. Eventually the nuclear equaiton of state of the
inner core stiffens and halts the collapse. Due to inertia the core overshoots
the new equlibrium size and bounces back into the still infalling material.
This laucnhes a shock that travels outwards into the acreting material.

CoreCollapseControl contains four main sub-parts: (1) phase detection, (2)
output parameter steering, (3) mesh control, (4) object tracking.

\subsection{Phase detection and output parameter steering}
CoreCollapseControl defines four stages ``collapse'', ``prebounce'',
``bounce'' and ``preBH''. 
\begin{description}
\item[collapse] is the initial stage of the simulation in which matter is
acreting onto the inner core. Unless \texttt{force\_postbounce} is in effect,
CoreCollapseControl.
\item[prebounce] is entered once the maximum density on the grid exceeds
\texttt{prebounce\_rho}. Upon entering this stage CoreCollapseControl steers
ooutput frequency and criterion parameters using \texttt{preb\_*} set of
values.
\todo{clean up this code. it does not currently seem to have very predictable
behaviour for 1d/2/3d output}
\item[bounce] stage is detected if either \texttt{bounc\_ciretion} is set to
``density'' and the maximum density on the grid exceeds \texttt{bounce\_rho} or
if \texttt{bounce\_criterion} is set to ``entropy'' and the maximum entropy on
the grid exceeds \texttt{bounce\_entropy}. Not output parameters are steered
however CoreCollapseControl records the time bounce occured in the grid scalar
\texttt{bouncetime}.
\item[preBH] proceeds in two stages. First, once the minimum lapse on the grid
drops below \texttt{preBH\_alpA} the output frequncy of CarpetIOHDF5 is steered
to \texttt{preBH\_out3Dhdf5\_every}, second once the minimum lapse on the grid
drops below \texttt{preBH\_alpB} AHFinderDirect's \texttt{find\_every}
parameter is steered to begin looking for apparent horizons.
\end{description}

Transitioning into any of the later stages is recorded and CoreCollapseControl
takes care not to drop into an earlier stage once a later one has been
reached. Curerntly CoreCollapseControl reverts to the ``collapse'' stage after
recovering from a checkpoint.

\todo{it should not do so. It shoud only re-examine its triggers when they
were changed. It should also most likely use AEIThorns::Trigger to check its
conditions}

\subsection{Progressive mesh refinement}
CoreCollapseControl implements progressive mesh refinement of the central
region of the simulation. CoreCollapseControl defines a list of density
thresholds \texttt{rho\_max\_list} and enables one more level of mesh
refinement whenever the maximum density on the grid exceeds the current
threshold level for the first time. A switch-off time
\texttt{switch\_off\_refinement\_after\_time} can be used to turn off all mesh
refinemnt after a given time.

\subsection{Object tracking}
CoreCollapseControl can track up to 10 fragments by following the motion of
high density regions. Given a set of previous fragment locations, the fragment
positions are updated by searching for the highest density point within
\texttt{search\_radius\_around\_old\_maximum}. CoreCollapseControl restricts
candidate points to have densiyt higher than \texttt{rho\_max\_threshold}, and
enforces a minimum separation of \texttt{min\_separation\_between\_maxima}
between each fragment location. Optionally the centroid position of
SphericalSurfaces can be set so that the fragment location can be used eg. as
initial guesses for AHFinderDirect. CoreCollapseControl provides options to
apply its progressive mesh refinement method to each of the tracked fragments.

\section{Using This Thorn}
To use this thorn you will have to activate HydroBase, ADMBase,
SphericalSurface and CarpetRegrid2. All other thorns whose parameters can be
steered are detected at runtime.

\subsection{Obtaining This Thorn}
Currently the thron is part of the Zelmani arrangement. \todo{RH: this must of
course change}

\subsection{Basic Usage}
The typical usage case is to add more and more refinement levels to a collaspe
simulations as the star collapses. This can be achieved by setting
\texttt{handle\_PMR} to true and providing a list of densities (in g cm$^3$)
in \texttt{rho\_max\_list}.

\subsection{Special Behaviour}
After recovering from a checkpoint CoreCollapseControl will re-exame the grid
variables to decide which stage it is in. \todo{RH: this needs to change}

\subsection{Interaction With Other Thorns}
CoreCollapseControl steers parameters in 
CarpetIOScalar, CarpetIOASCII,  CarpetIOHDF5, 
ZelmaniQuadWaveExtract, InnerCore, ZerilliEF, WaveExtract, NPScalars,
InnerCore, AHFinderDirect. It changes grid scalars in SphericalSurface and
CarpetRegrid2.

\subsection{Examples}
\begin{verbatim}
# set up grid structure for CoreCollapseControl to manage
CarpetRegrid2::num_centres  = 1
CarpetRegrid2::num_levels_1 = 1
CarpetRegrid2::position_x_1 = 0
CarpetRegrid2::position_y_1 = 0
CarpetRegrid2::position_z_1 = 0

CarpetRegrid2::radius_1[1]   = 192.0
CarpetRegrid2::radius_1[2]   = 144.0
CarpetRegrid2::radius_1[3]   = 98.4
CarpetRegrid2::radius_1[4]   = 40.0
CarpetRegrid2::radius_1[5]   = 12.0
CarpetRegrid2::radius_1[6]  =  6.0
CarpetRegrid2::radius_1[7]  =  3.0
CarpetRegrid2::radius_1[8]  =  2.0

# list of densities controlling when new levels are activated
CoreCollapseControl::handle_PMR = yes
CoreCollapseControl::rho_max_list[0] = 8.0e10
CoreCollapseControl::rho_max_list[1] = 3.2e11
CoreCollapseControl::rho_max_list[2] = 1.28e12
CoreCollapseControl::rho_max_list[3] = 5.12e12
CoreCollapseControl::rho_max_list[4] = 2.048e13
CoreCollapseControl::rho_max_list[5] = 3.0e15
CoreCollapseControl::rho_max_list[6] = 5.0e15
CoreCollapseControl::rho_max_list[7] = 9.0e88
CoreCollapseControl::check_every = 256
Corecollapsecontrol::rho_max_every = 256

# control output frequency depending on stage we are in
CoreCollapseControl::output_control = yes
CoreCollapseControl::prebounce_rho = 1.0e13
CoreCollapseControl::preb_out2D_every = 256
CoreCollapseControl::preb_out1D_every = 2048
CoreCollapseControl::preb_out0D_every = 256
CoreCollapseControl::preb_out3Dhdf5_every = -1
CoreCollapseControl::preb_checkpoint_every = 2048
CoreCollapseControl::preb_outscalar_every = 256
CoreCollapseControl::preb_waves_every = 64

CoreCollapseControl::preBH_alpA = 0.4e0
CoreCollapseControl::preBH_alpB = 0.4e0
CoreCollapseControl::preBH_AH_every = 64
CoreCollapseControl::preBH_out3Dhdf5_every = -1
\end{verbatim}

\subsection{Support and Feedback}
Support is available through the Einstein Toolkit mailing list
\url{users@einsteintoolkit.org}. Feedback can be provided using the same
channel.

\section{History}
Frist public version.

\subsection{Acknowledgements}
Anyone?

%\begin{thebibliography}{9}
%
%\end{thebibliography}

% Do not delete next line
% END CACTUS THORNGUIDE

\end{document}
