% *======================================================================*
%  Cactus Thorn template for ThornGuide documentation
%  Author: Ian Kelley
%  Date: Sun Jun 02, 2002
%  $Header$
%
%  Thorn documentation in the latex file doc/documentation.tex
%  will be included in ThornGuides built with the Cactus make system.
%  The scripts employed by the make system automatically include
%  pages about variables, parameters and scheduling parsed from the
%  relevant thorn CCL files.
%
%  This template contains guidelines which help to assure that your
%  documentation will be correctly added to ThornGuides. More
%  information is available in the Cactus UsersGuide.
%
%  Guidelines:
%   - Do not change anything before the line
%       % START CACTUS THORNGUIDE",
%     except for filling in the title, author, date, etc. fields.
%        - Each of these fields should only be on ONE line.
%        - Author names should be separated with a \\ or a comma.
%   - You can define your own macros, but they must appear after
%     the START CACTUS THORNGUIDE line, and must not redefine standard
%     latex commands.
%   - To avoid name clashes with other thorns, 'labels', 'citations',
%     'references', and 'image' names should conform to the following
%     convention:
%       ARRANGEMENT_THORN_LABEL
%     For example, an image wave.eps in the arrangement CactusWave and
%     thorn WaveToyC should be renamed to CactusWave_WaveToyC_wave.eps
%   - Graphics should only be included using the graphicx package.
%     More specifically, with the "\includegraphics" command.  Do
%     not specify any graphic file extensions in your .tex file. This
%     will allow us to create a PDF version of the ThornGuide
%     via pdflatex.
%   - References should be included with the latex "\bibitem" command.
%   - Use \begin{abstract}...\end{abstract} instead of \abstract{...}
%   - Do not use \appendix, instead include any appendices you need as
%     standard sections.
%   - For the benefit of our Perl scripts, and for future extensions,
%     please use simple latex.
%
% *======================================================================*
%
% Example of including a graphic image:
%    \begin{figure}[ht]
% 	\begin{center}
%    	   \includegraphics[width=6cm]{MyArrangement_MyThorn_MyFigure}
% 	\end{center}
% 	\caption{Illustration of this and that}
% 	\label{MyArrangement_MyThorn_MyLabel}
%    \end{figure}
%
% Example of using a label:
%   \label{MyArrangement_MyThorn_MyLabel}
%
% Example of a citation:
%    \cite{MyArrangement_MyThorn_Author99}
%
% Example of including a reference
%   \bibitem{MyArrangement_MyThorn_Author99}
%   {J. Author, {\em The Title of the Book, Journal, or periodical}, 1 (1999),
%   1--16. {\tt http://www.nowhere.com/}}
%
% *======================================================================*

% If you are using CVS use this line to give version information
% $Header$

\documentclass{article}

% Use the Cactus ThornGuide style file
% (Automatically used from Cactus distribution, if you have a
%  thorn without the Cactus Flesh download this from the Cactus
%  homepage at www.cactuscode.org)
\usepackage{../../../../doc/latex/cactus}

\begin{document}

% The author of the documentation
\author{Christian D. Ott \textless cott@tapir.caltech.edu\textgreater, \\
Erik Schnetter \textless eschnetter@perimeterinstitute.ca\textgreater}

% The title of the document (not necessarily the name of the Thorn)
\title{EOS\_Omni}

% the date your document was last changed, if your document is in CVS,
% please use:
%    \date{$ $Date: 2004-01-07 12:12:39 -0800 (Wed, 07 Jan 2004) $ $}
\date{February 24, 2013}

\maketitle

% Do not delete next line
% START CACTUS THORNGUIDE

% Add all definitions used in this documentation here
%   \def\mydef etc

\newenvironment{equationarray}
{\arraycolsep 0.14 em
\begin{eqnarray}}
{\end{eqnarray}}

\newenvironment{equationarray*}
{\arraycolsep 0.14 em
\begin{eqnarray*}}
{\end{eqnarray*}}

\begin{abstract}
  \noindent This thorn provides a unified EOS (Equation Of State)
  interface and implements multiple analytic EOS, and also provides table
  reader and interpolation routines for finite-temperature
  microphysical EOS available from
    \url{http://www.stellarcollapse.org}. In addition cold and barotropic
  tabulated EOS are provided for use in initial data thorns. Currently, the
  implemented analytic EOS are the polytropic EOS, the gamma-law EOS, and a
  hybrid EOS consisting of a 2-piece piecewise-polytrope with a thermal,
  gamma-law component.
\end{abstract}



\section{Introduction}

Equations of State (EOS) are crucial for hydrodynamics and hydro
codes (as well as other codes needing/providing microphysics)
are closely coupled to EOS and call EOS routines many times for
each grid point during the calculation of a time update.

\texttt{EOS\_Omni} is presently coded for cold and hot EOS, including
those based on microphysical models. It does currently assume nuclear
statistical equilibrium (NSE) with rest-mass density $\rho$, specific
internal energy $\epsilon$ (or temperature $T$), and electron fraction
$Y_e$ being the independent variables. \texttt{EOS\_Omni} can be
called on arrays or on single grid points.



\section{Units}

This thorn uses \emph{solar} units where $c = G = M_{\odot} = 1$.
Temperatures are measured in MeV.

\section{Using This Thorn}

\subsection{Basic Usage}

\texttt{EOS\_Omni} works via the aliased-function interface, and
EOS functions to be used must be declared in \texttt{interface.ccl}.
Here is an example \texttt{interface.ccl} entry:
\begin{verbatim}
void FUNCTION EOS_Omni_press(CCTK_INT IN eoskey,         \
                             CCTK_INT IN havetemp,       \
                             CCTK_REAL IN rf_precision,  \
                             CCTK_INT IN npoints,        \
                             CCTK_REAL IN ARRAY rho,     \
                             CCTK_REAL INOUT ARRAY eps,  \
                             CCTK_REAL INOUT ARRAY temp, \
                             CCTK_REAL IN ARRAY ye,      \
                             CCTK_REAL OUT ARRAY press   \
                             CCTK_INT OUT ARRAY keyerr,  \
                             CCTK_INT OUT anyerr)
\end{verbatim}

Here, 
\begin{itemize}
\item \texttt{eoskey} is the type of EOS to be used in this call.
  \begin{itemize}
    \item \texttt{eoskey = 1}: Polytropic EOS
    \item \texttt{eoskey = 2}: Gamma-Law EOS
    \item \texttt{eoskey = 3}: Hybrid EOS (2 Polys, 1 Gamma-Law), used
      for stellar core collapse simulations.
    \item \texttt{eoskey = 4}: Finite-temperature microphysical EOS
    \item \texttt{eoskey = 5}: Cold tabulated EOS with Gamma-Law
    \item \texttt{eoskey = 6}: Tabulated barotropic EOS
  \end{itemize}
\item \texttt{havetemp} determines whether the EOS is to be called as
  a function of $(\rho,\epsilon,Y_e)$ (\texttt{havetemp = 0}), or as a
  function of $(\rho,T,Y_e)$ (\texttt{havetemp = 1}).
  \texttt{havetemp = 0} is the method of choice for analytic EOS during
  evolution, but at the initial data stage one may need to set
  \texttt{havetemp = 1} (with $T=0$) to obtain initial values for
  $\epsilon$.  In the case of a finite-temperature EOS (that usually
  is a function of $(\rho,T,Y_e)$), a call with \texttt{havetemp = 0}
  will first lead to the solution of $T(\rho,\epsilon,Y_e$) via a
  Newton-Raphson-type iteration (using the supplied value of $T$ as
  the starting point) and will then calculate the requested dependent
  variable as a function of $X=X(\rho,T,Y_e)$. Both $X$ and the updated
  $T$ are returned.

\item \texttt{npoints} tells the EOS how many data points are passed
  in,
\item \texttt{rho,eps,temp,ye,press} are obvious,
\item \texttt{rf\_precision} is a real number telling the root finding
  algorithm (for finding $T(\rho,\epsilon,Y_e)$) at what relative error
  to terminate the iteration. $10^{-10}$ is a good value.
\item \texttt{keyerr} is an array (with $n$ entries for $n$ data
  points) containing error codes (relevant only for tabulated
  microphysical EOS),
\item \texttt{anyerr} is an integer $>0$ in case
  any error occured.
\end{itemize}

\subsection{Parameter Settings}

Many hydro codes require a fallback EOS in case something goes
wrong. This is also true for the Einstein Toolkit GR hydro code
\texttt{EinsteinEvolve/GRHydro}. If you want to use \texttt{EOS\_Omni} with
\texttt{GRHydro}, you must set parameters for the EOS of your choice and,
\emph{in addition}, the following parameters must be set to sensible values:

\begin{verbatim}
eos_omni::poly_gamma 
eos_omni::poly_gamma_initial
eos_omni::poly_k 
\end{verbatim}

The only non-obvious parameter here is \texttt{poly\_gamma\_initial}.
In most simulations it should not be set.
In simulations that are run with a different adiabatic index than
what was used to set up the initial data, \texttt{poly\_gamma} should
be the evolution value, and \texttt{poly\_gamma\_initial} should be the
initial data value. \texttt{EOS\_Omni} then rescales \texttt{poly\_k} such
that the \emph{cgs} value of \texttt{poly\_k} is the same for initial data and
evolution. Since the units of \texttt{poly\_k}
($[K] = \frac{\rm N}{\rm m^2} \left(\frac{\rm m^3}{\rm kg}\right)^{\Gamma}$)
depend on \texttt{poly\_gamma} this is not fully trivial.

Check \texttt{param.ccl} for parameters for the other EOS\@.



\section{Equations of State Details}

\subsection{Polytropic}

The \texttt{poly} EOS is a \emph{polytropic} equation of state, which
does not allow for changes in entropy:
\begin{eqnarray}
  p & = & K \rho^\gamma
\end{eqnarray}
where $p$ is the pressure, $\rho$ the density, $K$ the polytropic
constant set via \texttt{poly\_k}, and $\gamma$ is the adiabatic index
set via \texttt{poly\_gamma}.

If the internal energy $\epsilon$ is to be ``calculated from the
temperature'' (\texttt{havetemp = 1}), then this is done using the
relation
\begin{eqnarray}
  \epsilon & = & \frac{K}{\gamma-1} \rho^{\gamma-1}
\label{eq:polyeps}
\end{eqnarray}
(which actually ignores the temperature).

Note: This polytropic EOS is also used as fall-back when other
EOS fail.

\subsection{Gamma-Law}

The \texttt{gl} EOS is a \emph{gamma-law} equation of state,
corresponding to an ideal gas:
\begin{eqnarray}
  p & = & (\gamma-1) \rho \epsilon
\end{eqnarray}
where $p$ is the pressure, $\rho$ the density, $\epsilon$ the internal
energy, and $\gamma$ is the adiabatic index set via
\texttt{gl\_gamma}.

At the initial data stage, it may be necessary to set up initial
values for $\epsilon$. For this, the \texttt{gl} EOS 
implements equation (\ref{eq:polyeps}) just like the \texttt{poly}
EOS and the parameters \texttt{poly\_gamma\_ini} and 
\texttt{gl\_k} must be set for this.



%%% The parameter \texttt{poly\_gamma\_ini} (not \texttt{gl\_gamma\_ini}!)
%%% influences the definition of this EOS via unit conversions: $K$ is
%%% first converted from solar units to cgs units using
%%% \texttt{poly\_gamma\_ini}, and then converted back to solar units
%%% using \texttt{gl\_gamma}. \textbf{TODO: confirm this.} Note that the
%%% parameter \texttt{gl\_gamma\_ini} is actually unused. \textbf{TODO:
%%%   confirm this.} Contrary to what the parameter name suggests, this
%%% EOS thorn makes no distinction between initial data setup and time
%%% evolution; if such a distinction is required, then it must be
%%% implemented outside of this thorn.
%%% If the internal energy $\epsilon$ is to be ``calculated from the
%%% temperature'' (\texttt{havetemp = 1}), then this is done using the
%%% relation
%%% \begin{eqnarray}
%%%   \epsilon & = & \frac{K}{\gamma-1} rho^{\gamma-1}
%%% \end{eqnarray}
%%% (which actually ignores the temperature). $K$ the polytropic constant
%%% set via \texttt{gl\_k}. This is the same relation as for the
%%% polytropic equation of state above.

\subsection{Hybrid}

The hybrid EOS was introduced by \cite{janka:93} for use in simplified
simulations of stellar collapse to mimic (1) the stiffening of the
nuclear EOS at nuclear density and (2) to include thermal pressure in
the postbounce phase.  It consists of two polytropes characterized by
($K_1$, $\gamma_1$) and ($K_2$, $\gamma_2$) and a thermal $\gamma-$law
component described by $\gamma_{\mathrm{th}}$.  Polytrope 1 is soft and
describes a gas of relativistic degenerate electrons with $\gamma_1
\approx 4/3$.  It is used below nuclear density ($\rho_{\mathrm{nuc}}
\approx 2\times10^{14}\,\mathrm{g\,cm}^{-3}$), and is smoothly matched to
polytrope 2 which applies above $\rho_{\mathrm{nuc}}$, is stiff, and
models the repulsive core of the strong force above nuclear density
($\gamma_2 \gtrsim 2.5$).  $K_2$ is completely determined by
$P_1(\rho_{\mathrm{nuc}}) = P_2(\rho_{\mathrm{nuc}})$ and $K_1, \gamma_1,$
and $\gamma_2$. The full functional form of the EOS
$P=P(\rho,\epsilon)$ with the thermal component (which takes into
account shock heating) is given by
\begin{equationarray}
  P & = & \frac{\gamma - \gamma_{\rm th}}{\gamma - 1}
  K \rho_{\rm nuc}^{\gamma_1 - \gamma}
  \rho^{\gamma} - \frac{(\gamma_{\rm th} - 1) (\gamma - \gamma_1)}
  {(\gamma_1 - 1) (\gamma_2 - 1)}
  K \rho_{\rm nuc}^{\gamma_1 - 1} \rho 
  + (\gamma_{\rm th} - 1) \rho \epsilon\,.
  \label{eq:hybrid_eos}
\end{equationarray}%

The \texttt{EOS\_Omni} parameters for the hybrid EOS are the following:

\begin{tabular}{ll}
\texttt{hybrid\_gamma1}     & $\gamma_1$,  $\gamma_1 = 1.325$ is an appropriate choice.\\
\texttt{hybrid\_gamma2}     & $\gamma_2$,  $\gamma_2 = 2.5$ is an appropriate choice.\\
\texttt{hybrid\_gamma\_th}   & $\gamma_{\mathrm{th}}$, perhaps $1.5$.\\
\texttt{hybrid\_k1}         & $K_1$, $0.4640517$ in solar units for relativistic degenerate e$^{-}$.\\
\texttt{hybrid\_rho\_nuc}    & nuclear density, standard is $3.238607\times 10^{-4}$ in solar units.
\end{tabular}





\subsection{Finite-Temperature Nuclear EOS}

Complex microphysical finite-temperature equations of state come
usually in tabulated form. {\tt EOS\_Omni} comes with routines
provided as part of the \texttt{nuc\_eos} package described in
\cite{oconnor:10} and available at \\ {\tt
  http://www.stellarcollapse.org}. A variety of EOS tables for
application in high-density astrophysical situations (i.e. in stellar
collapse or in compact star mergers) are also available from there
in HDF5 format.

The parameters controlling the finite-temperature nuclear EOS
are the following:

\begin{tabular}{lll}
\texttt{nuceos\_read\_table}& BOOLEAN & Set to {\tt yes} to read table.\\
\texttt{do\_energy\_shift}& BOOLEAN & Set to {\tt yes} to subtract the energy shift\\ 
&& stored in the table to get correctly normalized $\epsilon$.\\
\texttt{nuceos\_table\_name}& STRING & Path/Name of the table file.
\end{tabular}

\subsection{Cold Tabulated Nuclear EOS with Gamma Law}

Many equations of state for neutron stars are generated under the
assumption of zero temperature. This is perfectly appropriate for cold
old neutron stars. In simulations of binary mergers, however, shocks
will drive the temperature up, adding a thermal pressure component,
which can be accounted for approximately with a Gamma-Law:
$P_\mathrm{th} = (\Gamma_\mathrm{th} - 1)\rho\epsilon_\mathrm{th}$\,.

{\tt EOS\_Omni} implements such an equation of state. It reads in an
ASCII EOS table (see subdirector {\tt tables} for an example table for
the SLy EOS \cite{douchin:01,haensel:04}, which was generated
according to the prescription in \cite{shibata:05,corvino:10}). All
EOS parameters are read from the ASCII file, which has the following
format:

\begin{verbatim}
EoSType = Tabulated
Nrho = 600    NYe = 1     NT = 1
RhoMin = 1e-09  RhoMax = 0.01
HeatCapacityE = 1
GammaTh = 2
Kappa = 1
RhoSpacing = Log
    1.57940636422747e-03    1.38773826035349e+00    2.62139412738900e-02
[...]
    2.81804006881059e+00    6.89967555695907e-01    8.30537612378975e-01
\end{verbatim}

The header completely determines the range in baryon rest mass density
(in $c=G=M_\odot=1$ units), gives the number of zones (currently only
{\tt NYe = 1}, {\tt NT = 1}, {\tt HeatCapacityE = 1}, and {\tt
  RhoSpacing = Log} are supported). {\tt GammaTh} is the
$\Gamma_\mathrm{th}$ of the thermal gamma law. The tabulated columns
are $\epsilon$, $\Gamma$, $c_s$ (the speed of sound of the cold
component of the EOS). The pressure is obtained via $P= \kappa
\rho^\Gamma$, where $\kappa$ is the {\tt Kappa} scaling parameter.

Generally, $P = P(\rho,\epsilon)$ in this EOS, but note that
$\epsilon_\mathrm{th} = \epsilon - \epsilon_\mathrm{cold}$. {\tt
  EOS\_Omni} uses linear interpolation to first find $P(\rho)$ and
$\epsilon_\mathrm{cold}(\rho)$ and then computes the thermal component
analytically.


\begin{tabular}{lll}
\texttt{coldeos\_read\_table}& BOOLEAN & Set to {\tt yes} to read table.\\
\texttt{coldeos\_use\_thermal\_gamma\_law}& BOOLEAN & Set to {\tt yes} to use the thermal
gamma law (default).\\
\texttt{coldeos\_table\_name}& STRING & Path/Name of the table file.\\
\end{tabular}


\section{Converting Old Parameter Files}

If you have a parameter file that uses the previous EOS interface in
Cactus, you will have to convert it so that it runs with
\texttt{EOS\_Omni}. The following describes a set of simple rules for
this conversion.

\begin{enumerate}
\item Add \texttt{EOS\_Omni} to the thorn list. You can then remove
  all other \texttt{EOS\_*} thorns from the thorn list (or you can
  leave them in; they are unused).
\item Activate \texttt{EOS\_Omni} in the parameter file: Add
  \texttt{EOS\_Omni} to your active thorns, and do not activate any
  other \texttt{EOS\_*} thorns.
\item Translate all EOS parameters according to table
  \ref{tab:paramconv}.
\item All thorns using this EOS interface will have a parameter that
  determines which EOS to use, typically via a string/keyword
  parameter specifying an EOS name. Convert these names using table
  \ref{tab:eosnames}.
\end{enumerate}

\begin{table}[h]
  \begin{tabular}{ll|ll}
    Old Parameter & Old Value & New Parameter & New Value\\
    \hline
    EOS\_Polytrope::eos\_gamma & &
    EOS\_Omni::poly\_gamma &
    \\
    EOS\_Polytrope::eos\_k & &
    EOS\_Omni::poly\_k &
    \\
    EOS\_Polytrope::use\_cgs & yes &
    --- &
    \\
    EOS\_Polytrope::use\_cgs & no &
    unsupported &
    \\
    EOS\_Polytrope::gamma\_ini & &
    EOS\_Omni::poly\_gamma\_ini &
    \\
    EOS\_Ideal\_Fluid::eos\_ideal\_fluid\_gamma & &
    EOS\_Omni::gl\_gamma &
    \\
    \textbf{TODO: complete this table}
  \end{tabular}
  \caption{Parameter conversion table}
  \label{tab:paramconv}
\end{table}

\begin{table}[h]
  \begin{tabular}{llll}
    EOS    & Description                & Old Name & New Name \\\hline
    poly   & polytropic                 & ???      & 2D\_Polytrope \\
    gl     & gamma-law                  & ???      & Ideal\_Fluid \\
    hybrid & hybrid                     & ???      & Hybrid \\
    nuc    & finite-temperature nuclear & ???      & nuc\_eos
  \end{tabular}
  \caption{EOS name conversion table}
  \label{tab:eosnames}
\end{table}



\begin{thebibliography}{9}
\bibitem{janka:93} Janka, H.-T., Zwerger, T., \& Moenchmeyer, R.\ 1993, Astron. Astrophys., 268, 360 

\bibitem{oconnor:10} O'Connor, E., \& Ott, C.~D.\ 2010, Class. Quantum Grav., 27, 114103 

\bibitem{douchin:01} Douchin, F., \& Haensel, P.\ 2001, Astron.~Astrophys., 380, 151 

\bibitem{haensel:04} Haensel, P., \& Potekhin, A.~Y.\ 2004, Astron.~Astrophys., 428, 191 

\bibitem{shibata:05} Shibata, M., Taniguchi, K., \& Ury{\= u}, K.\ 2005, Phys.~Rev.~D, 71, 084021 

\bibitem{corvino:10} Corvino, G., Rezzolla, L., Bernuzzi, S., De Pietri, R., 
\& Giacomazzo, B.\ 2010, Class. Quantum Grav., 27, 114104 













\end{thebibliography}

% Do not delete next line
% END CACTUS THORNGUIDE

\end{document}
